\documentclass[journal]{IEEEtran}
\usepackage[a5paper, margin=10mm]{geometry}
%\usepackage{lmodern} % Ensure lmodern is loaded for pdflatex
\usepackage{tfrupee} % Include tfrupee package


\setlength{\headheight}{1cm} % Set the height of the header box
\setlength{\headsep}{0mm}     % Set the distance between the header box and the top of the text


%\usepackage[a5paper, top=10mm, bottom=10mm, left=10mm, right=10mm]{geometry}

%
\usepackage{gvv-book}
\usepackage{gvv}
%\setlength{\intextsep}{10pt} % Space between text and floats

\makeindex

\begin{document}
\bibliographystyle{IEEEtran}
\onecolumn


\title{
	%\begin{flushleft}
	\begin{center}
	%MATRICES \\ In Geometry
	C Programming in Middle School
%	Progressions
	\\
\rule{0.4\columnwidth}{0.4pt}
%\end{flushleft}
\end{center}
}
\author{
\vspace{11cm}
	%\begin{flushleft}
	\begin{center}
\includegraphics[width=0.2\columnwidth]{figs/logo.jpg}
\\
		{\huge	G. V. V. Sharma}\\Associate Professor,\\Department of Electrical Engineering, \\ IIT Hyderabad
	\end{center}
	%\end{flushleft}
%\IEEEpubid{\makebox[\columnwidth]{978-1-7281-5966-1/20/\$31.00 ©2020 IEEE \hfill} \hspace{\columnsep}\makebox[\columnwidth]{ }}
}
\maketitle

\newpage
\section*{About this Book}

This book introduces C programming for middle school children based on the
 NCERT mathematics textbook of Class 7.  

There is no copyright, so readers are free to print and share.  

This book is dedicated to my Hindi teacher in middle school, Shri Mandavi.
\begin{flushright}
\today
\end{flushright}
Github: https://github.com/gadepall/cprog
		\\
License: https://creativecommons.org/licenses/by-sa/3.0/
\\
and
\\
https://www.gnu.org/licenses/fdl-1.3.en.html

\newpage


\tableofcontents

\newpage
%\twocolumn
\onecolumn


%\renewcommand{\theequation}{\theenumi}
\numberwithin{equation}{enumi}
%\numberwithin{figure}{enumi}
%\numberwithin{figure}{section}
%\numberwithin{figure}{subsection}
\renewcommand{\thefigure}{\theenumi}
\renewcommand{\thetable}{\theenumi}

\section{Integers}
\subsection{Formulae}
\begin{enumerate}[label=\thesubsection.\arabic*, ref=\thesubsection.\theenumi]
\item Do the following addition through a C program
	$$17+23$$
	\\
	\solution
	\lstinputlisting{codes/integer/add.c}
\item Do the following subtraction through a C program
$$7-9$$
	\\
	\solution
	\lstinputlisting{codes/integer/sub.c}
\item Mulitply the following through a C program
	$$4\times \brak{-8}$$
	\\
	\solution
	\lstinputlisting{codes/integer/mul.c}
\item Perform the following division
	$$\brak{-100}\div 5$$
	\\
	\solution
	\lstinputlisting{codes/integer/div.c}
\end{enumerate}

\subsection{NCERT}
Compute the following
\begin{enumerate}[label=\thesubsection.\arabic*, ref=\thesubsection.\theenumi]
	\begin{multicols}{4}
\item $\brak{-75}+18$
\item $19+\brak{-25}$
\item $27+\brak{-27}$
\item $\brak{-20}+0$
\item $\brak{-35}+\brak{-10}$
\item $\brak{-10}+3$
\item $17-(-21)$
\item $8\times(-2)$
\item $3\times(-7)$
\item $10\times(-1)$
\item $6\times(-19)$
\item $12\times(-32)$
\item $7\times(-22)$
\item $15\times(-16)$
\item $21\times(-32)$
\item $(-42)\times 12$
\item $(-55)\times 15$
\item $(-5)\times \brak{-6}$
\item $\brak{-6}\times(-7)$
\item $3\times(-1)$
\item $\brak{-1}\times 225$
\item $\brak{-21}\times(-30)$
\item $\brak{-316}\times(-1)$
\item	$\brak{-81}\div 9$
\item	$\brak{-75}\div 5$
\item	$\brak{-32}\div 2$
\item	$125\div \brak{-25}$
\item	$80\div \brak{-5}$
\item	$64\div \brak{-16}$
\item	$\brak{-30}\div 10$
\item	$50\div \brak{-5}$
\item	$\brak{-36}\div \brak{-9}$
\item	$\brak{-49}\div \brak{-49}$
\end{multicols}
\end{enumerate}
\begin{enumerate}[label=\thesubsection.\arabic*, ref=\thesubsection.\theenumi,resume*]
	\begin{multicols}{2}
\item	$13\div \sbrak{\brak{-2}+1}$
\item	$\brak{-31}\div \sbrak{\brak{-30}+\brak{-1}}$
\item	$\sbrak{\brak{-36}\div 12}\div \brak{3}$
\item	$\sbrak{\brak{-6}+5}\div \sbrak{\brak{-2}+1}$
	\end{multicols}
\end{enumerate}
Fill in the blanks
\begin{enumerate}[label=\thesubsection.\arabic*, ref=\thesubsection.\theenumi,resume*]
	\begin{multicols}{2}
		\item	$20 \div \rule{1cm}{0.1pt}=-2$
		\item	$\rule{1cm}{0.1pt}\div 4=-3$
	\end{multicols}
\end{enumerate}
Find the values of the following expressions for $x = 2$. 
	\begin{multicols}{3}
\begin{enumerate}[label=\thesubsection.\arabic*, ref=\thesubsection.\theenumi,resume*]
\item $x + 4$
\item  $4x – 3$ 
\item $ x – 2$
\item $ 3x – 5 $
\item $ 9 – 5x $
\item $ x + 7 + 4 (x – 5)$
\item $ 3 (x + 2) + 5x – 7 $
\item $ 6x + 5 (x – 2) $
\item $ 4(2x – 1) + 3x + 11$
\end{enumerate}
\end{multicols}
If $x = – 2$, find the value of
	\begin{multicols}{2}
\begin{enumerate}[label=\thesubsection.\arabic*, ref=\thesubsection.\theenumi,resume*]
\item $5x – 2$
\item $4p + 7$
\end{enumerate}
\end{multicols}
Find the value of the following expressions for $a = 3, b = 2$. 
	\begin{multicols}{2}
\begin{enumerate}[label=\thesubsection.\arabic*, ref=\thesubsection.\theenumi,resume*]
\item $a + b$				
\item $ 7a – 4b $
\end{enumerate}
\end{multicols}
Find the value of the following expressions, when $x = –1$ 
\begin{enumerate}[label=\thesubsection.\arabic*, ref=\thesubsection.\theenumi,resume*]
\item $ 2x – 7$
\item $ – x + 2$ 
\end{enumerate}
When $a = 0, b = – 1$, find the value of the given expressions
\begin{enumerate}[label=\thesubsection.\arabic*, ref=\thesubsection.\theenumi,resume*]
\item $2a + 2b$
\end{enumerate}
Simplify these expressions and find their values if $x = 3, a = – 1, b = – 2$. 
	\begin{multicols}{4}
\begin{enumerate}[label=\thesubsection.\arabic*, ref=\thesubsection.\theenumi,resume*]
\item $ 2x +4 $
\item $6 - 4x$
\item $6 – 5a $
\item $6 – 8b $
\item $3a – 2b-9 $
\end{enumerate}
\end{multicols}
\begin{enumerate}[label=\thesubsection.\arabic*, ref=\thesubsection.\theenumi,resume*]
	\item In a test (+5) marks are given for every correct answer and (-2) marks for every incorrect answer.  
		\begin{enumerate}
			\item Radhika answered all the questions and scored 30 marks though she got 10 correct answers.
			\item Jay also answered all the questions and scored (-12) marks though he got 4 correct answers.  How many incorrect answers had they  attempted?
		\end{enumerate}
	\item A shopkeeper earns a profit of \rupee 1 by selling one pen and incurs a loss of 40 paise per pencil while selling pencils of her old stock.  
		\begin{enumerate}
			\item In a particular month she incurs a loss of \rupee 5.  In this period she sold 45 pens.  How many pencils did she sell in this period?
			\item In the next month she earns neither profit nor loss.  If she sold 70 pens, how many pencils did she sell?
		\end{enumerate}
			\item The temperature at 12 noon was 10\degree C above zero. If it decreases at the rate of 2\degree C per hour unitl midnight, at what time would the temperature be 8\degree C below zero? 
	\item In a class test (+3) marks are given for every correct answer and (-2) marks for every incorrect answer and no marks for not attempting any question.   
		\begin{enumerate}
			\item Radhika scored 20 marks.  If she got 12 correct answers, how many questions has she attempted incorrectly?
			\item Mohini scores -5 marks in this test, though she has got 7 correct answers.   How many questions has she attempted incorrectly?
		\end{enumerate}
	\item An elevator descends a mine shaft at the rate of $6m/min$.  If the descent starts from $10m$ above the ground, how long will it take to reach $-350m$.
	\item What is the measure of the complement of each of the following angles? 
		\begin{enumerate}
			\begin{multicols}{4}
	\item 45\degree
\item 65\degree 
\item 41\degree 
\item 54\degree 
			\end{multicols}
\end{enumerate}
\item What will be the measure of the supplement of each one of the following angles? 
		\begin{enumerate}
			\begin{multicols}{4}
\item 100\degree
\item  90\degree 
\item  55\degree 
\item  125\degree
			\end{multicols}
\end{enumerate}
\item An exterior angle of a triangle is of measure $70\degree$ and one of its interior opposite angles is of measure 25\degree. Find the measure of the other interior opposite angle.
\item The two interior opposite angles of an exterior angle of a triangle are $60\degree$ and 80\degree. Find the measure of the exterior angle.
\item 	Two angles of a triangle are $30\degree$ and 80\degree. Find the third angle. 
\item  One of the angles of a triangle is $80\degree$ and the other two angles are equal. Find the measure of each of the equal angles.
\item  The three angles of a triangle are in the ratio 1:2:1. Find all the angles of the triangle. Classify the triangle in two different ways.
\item One of the sides and the corresponding height of a parallelogram are 4 cm and 3 cm respectively. Find the area of the parallelogram.
\item Find the height $x$ if the area of the parallelogram is 24 $cm^2$ and the base is 4 cm.
\item Find BC, if the area of the triangle ABC is 36 $cm^2$ and the height AD is 3 cm.
		\end{enumerate}

\section{Decimal Numbers}
\subsection{Formulae}
Find
\begin{enumerate}[label=\thesubsection.\arabic*,ref=\thesubsection.\theenumi,itemsep=1ex]
	\item $\frac{5}{3}+\frac{3}{5}$
		\\
		\solution
	\lstinputlisting{codes/deci/add.c}
	\item $\frac{2}{3}-\frac{3}{7}$
		\\
		\solution
	\lstinputlisting{codes/deci/sub.c}
	\item $5.6\times 1.4$
		\\
		\solution
	\lstinputlisting{codes/deci/mul.c}
	\item $ 37.8\div 1.4$
		\\
		\solution
	\lstinputlisting{codes/deci/div.c}
\end{enumerate}

\subsection{NCERT}
Find
\begin{enumerate}[label=\thesubsection.\arabic*,ref=\thesubsection.\theenumi,itemsep=1ex]
	\begin{multicols}{4}
%
	\item $\frac{2}{7}\times 3$
	\item $\frac{9}{7}\times 6$
	\item $\frac{1}{8}\times 3$
	\item $\frac{13}{11}\times 6$
	\item $\frac{2}{5}\times 2$
	\item $3\times 5\frac{1}{5}$
	\item $5\times 6\frac{3}{4}$
	\item $7\times 2\frac{1}{4}$
	\item $4\times 6\frac{1}{3}$
	\item $6\times 3\frac{1}{4}$
	\item $8\times 3\frac{2}{5}$
	\item $\frac{1}{2}\times \frac{1}{7}$
	\item $\frac{1}{5}\times \frac{1}{7}$
	\item $\frac{1}{3}\times \frac{4}{5}$
	\item $\frac{2}{3}\times \frac{1}{5}$
	\item $\frac{8}{3}\times \frac{4}{7}$
	\item $\frac{3}{4}\times \frac{2}{3}$
	\item $\frac{2}{3}\times 2\frac{2}{3}$
	\item $\frac{2}{7}\times \frac{7}{9}$
	\item $\frac{3}{8}\times \frac{6}{4}$
	\item $\frac{9}{5}\times \frac{3}{5}$
	\item $\frac{1}{3}\times \frac{15}{8}$
	\item $\frac{11}{2}\times \frac{3}{10}$
	\item $\frac{4}{5}\times \frac{12}{7}$
	\item $\frac{2}{5}\times 5\frac{1}{4}$
	\item $6\frac{2}{5}\times \frac{7}{9}$
	\item $\frac{3}{2}\times 5\frac{1}{3}$
	\item $\frac{5}{6}\times 2\frac{3}{7}$
	\item $3\frac{2}{5}\times \frac{4}{7}$
	\item $2\frac{3}{5}\times 3$
	\item $3\frac{4}{7}\times \frac{3}{5}$
	\item $\frac{2}{3}\times \rule{0.5cm}{0.1pt}=\frac{10}{30}$
	\item $\frac{3}{5}\times \rule{0.5cm}{0.1pt}=\frac{24}{75}$
	\item $7 \div \frac{2}{5}$
	\item $6 \div \frac{4}{7}$
	\item $2 \div \frac{8}{9}$
	\item $\frac{3}{5} \div \frac{1}{2}$
	\item $\frac{1}{2} \div \frac{3}{5}$
	\item $2\frac{1}{2} \div \frac{3}{5}$
	\item $5\frac{1}{6} \div \frac{9}{2}$
	\item $12 \div \frac{3}{4}$
	\item $14 \div \frac{5}{6}$
	\item $8 \div \frac{7}{3}$
	\item $4 \div \frac{8}{3}$
	\item $3 \div 2\frac{1}{3}$
	\item $5 \div 3\frac{4}{7}$
	\item $\frac{7}{3}\div 2$
	\item $\frac{4}{9}\div 5$ 
	\item $\frac{6}{13}\div 7$
	\item $4\frac{1}{3}\div 3$
	\item $3\frac{1}{2}\div 4$
	\item $4\frac{3}{7}\div 7$
	\item $\frac{2}{5} \div \frac{1}{2}$
	\item $\frac{4}{9} \div \frac{2}{3}$
	\item $\frac{3}{7} \div \frac{8}{7}$
	\item $2\frac{1}{3} \div \frac{3}{5}$
	\item $3\frac{1}{2} \div \frac{8}{3}$
	\item $\frac{2}{5} \div 1\frac{1}{2}$
	\item $3\frac{1}{5} \div 1\frac{2}{3}$
	\item $2\frac{1}{5} \div 1\frac{1}{5}$
	\item $0.2\times 6$
	\item $8\times 4.6$
	\item $2.71\times 5$
	\item $20.1\times 4$
	\item $0.05\times 7$
	\item $211.02\times 4$
	\item $2\times 0.86$
	\item $2.5\times 0.3$
	\item $0.1\times 51.7$
	\item $0.2\times 316.8$
	\item $1.3\times 3.1$
	\item $0.5\times 0.05$
	\item $11.2\times 0.15$
	\item $1.07\times 0.02$
	\item $10.05\times 1.05$
	\item $101.01\times 0.01$
	\item $100.01\times 1.1$
	\item $7.75\times 0.25$
	\item $42.8\times 0.02$
	\item $0.4 \div  2$
	\item $0.35 \div  5$
	\item $2.48 \div  4$
	\item $65.4 \div  6$
	\item $651.2 \div  4$
	\item $14.49 \div  7$
	\item $3.96  \div  4$
	\item $0.80 \div  5$
	\item $7 \div 3.5$
	\item $ 36\div 0.2$
	\item $3.25 \div 0.5$
	\item $ 30.94\div 0.7$
	\item $ 0.5\div 0.25$
	\item $ 7.75\div 0.25$
	\item $ 76.5\div 0.15$
	\item $ 2.73\div 1.3$
	\item $\frac{5}{4}+\frac{-11}{4}$
	\item $\frac{-9}{10}+\frac{22}{15}$
	\item $\frac{-3}{-11}+\frac{5}{9}$
	\item $\frac{-8}{19}+\frac{-2}{57}$
	\item $\frac{-2}{3}+{0}$
	\item $\frac{-13}{7}+\frac{6}{7}$
	\item $\frac{19}{5}+\frac{-7}{5}$
	\item $\frac{-5}{6}+\frac{-3}{11}$
	\item $-2\frac{1}{3}+4\frac{3}{5}$
	\item $\frac{7}{24}-\frac{17}{36}$
	\item $\frac{5}{63}-\frac{-6}{21}$
	\item $\frac{-6}{13}-\frac{-7}{15}$
	\item $\frac{-3}{8}-\frac{7}{11}$
	\item $-2\frac{1}{9}-{6}$
	\item $\frac{7}{9}-\frac{2}{5}$
	\item $2\frac{1}{5}-\frac{-1}{3}$
	\item $\frac{9}{2}\times \frac{-7}{4}$
	\item $\frac{3}{10}\times {-9}$
	\item $\frac{-6}{5}\times \frac{9}{11}$
	\item $\frac{3}{7}\times \frac{-2}{5}$
	\item $\frac{3}{11}\times \frac{2}{5}$
	\item $\frac{3}{-5}\times \frac{-5}{3}$
	\item ${-4}\div \frac{2}{3}$
	\item $\frac{-3}{5}\div {2}$
	\item $\frac{-4}{5}\div {-3}$
	\item $\frac{-1}{8}\div \frac{3}{4}$
	\item $\frac{-2}{13}\div \frac{1}{7}$
	\item $\frac{-7}{12}\div \frac{-2}{13}$
	\item $\frac{3}{13}\div \frac{-4}{65}$
	\item $\frac{-3}{5}\times 7\frac{}{}$
	\item $\frac{-6}{5}\times -2\frac{}{}$
	\item $\frac{-3}{4}\times \frac{1}{7}$
	\item $\frac{2}{3}\times \frac{-5}{9}$
	\item $\frac{2}{3}\times \frac{-7}{8}$
	\item $\frac{-6}{7}\times \frac{5}{7}$
	\end{multicols}
\end{enumerate}
Find
\begin{enumerate}[label=\thesubsection.\arabic*,ref=\thesubsection.\theenumi,resume*,itemsep=1ex]
	\begin{multicols}{2}
	\item $\frac{1}{2}$ of
		\begin{enumerate}
			\item $2\frac{3}{4}$
			\item $4\frac{2}{9}$
		\end{enumerate}
	\item $\frac{5}{8}$ of
		\begin{enumerate}
			\item $3\frac{5}{6}$
			\item $9\frac{2}{3}$
		\end{enumerate}
	\end{multicols}
\end{enumerate}
Find
\begin{enumerate}[label=\thesubsection.\arabic*,ref=\thesubsection.\theenumi,resume*]
	\begin{multicols}{2}
	\item $\frac{1}{4}$ of
		\begin{enumerate}
			\item $\frac{1}{4}$
			\item $\frac{3}{5}$
			\item $\frac{4}{3}$
		\end{enumerate}
	\item $\frac{1}{7}$ of
		\begin{enumerate}
			\item $\frac{2}{9}$
			\item $\frac{6}{5}$
			\item $\frac{3}{10}$
		\end{enumerate}
	\end{multicols}
\end{enumerate}
\begin{enumerate}[label=\thesubsection.\arabic*,ref=\thesubsection.\theenumi,resume*]
	\item  In a class of 40 students $\frac{1}{5}$ of the total number of students like to study English, 
$\frac{2}{5}$ of the total number like to study Mathematics and the remaining students like to study Science.
\begin{enumerate}
	\item How many students like to study English?
	\item How many students like to study Mathematics?
	\item How many students like to study Science?
\end{enumerate}
\item Vidya and Pratap went for a picnic.  Their mother gave them a water bottle that contained 5 litres of water.  Vidya consumed $\frac{2}{5}$ of the water.  Pratap consumed the remaining water.
	\begin{enumerate}
		\item How much water did Vidya drink?
		\item What fraction of the total quantity of water did Pratap drink?
	\end{enumerate}
\item Shaili plants 4 saplings in a row, in her garden.  The distance between two adjacent saplings is $\frac{3}{4}m$. Find the distance between the first and the last sapling.
\item Lipika reads a book for $1\frac{3}{4}$ hours everyday.  She reads the entire book in 6 days.  How many hours in all were required by her to read the book.
\item A car runs $16km$ using 1 litre of petrol.  How much distance will it cover using $2\frac{3}{4}$ litres of petrol.
\item The side of an equilateral triangle is $3.5cm$.  Find its perimeter.  
\item The length of a rectangle is $7.1cm$ and its breadth is $2.5cm$.  What is its area?
\item Find the area of a rectangle whose length is $5.7cm$ and breadth is $3cm$. 
\item A two wheeler covers a distance of $55.3km$ in one litre of petrol.  How much will it cover in 10 litres of petrol?
\item Savita was preparing a design to decorate her classroom.  She needed a few coloured strips of paper of length $1.9cm$ each.  She had a strip of coloured paper of length $9.5cm$.  How many pieces of the required length will she get out of this strip? 
\item Each side of a regular polygon is $2.5cm$ in length.  The perimeter of the polygon is $12.5cm$.  How many sides does the polygon have?
\item A car covers a distance of $89.1km$ in $2.2$ hours.  What is the average distance covered by it in 1 hour?
\item A vehicle covers a distance of $43.2km$ in in $2.4$ litres of petrol.  How much will it cover in one litre of petrol?
\item 	Mala has a collection of bangles. She has 20 gold bangles and 10 silver bangles. What is the percentage of bangles of each type? Can you put it in the tabular form?
\item 	Out of 25 children in a class, 15 are girls. What is the percentage of girls?
\item Out of 32 students, 8 are absent. What per cent of the students are absent? 
\item There are 25 radios, 16 of them are out of order. What per cent of radios are out of order?
\item  A shop has 500 items, out of which 5 are defective. What per cent are defective? 
\item  There are 120 voters, 90 of them voted yes. What per cent voted yes?
\item 	A survey of 40 children showed that 25\% liked playing football. How many children liked playing football?
\item	Rahul bought a sweater and saved  \rupee 200 when a discount of 25\% was given. What was the price of the sweater before the discount?
\item 	9 is 25\% of what number? 
\item 	75\% of what number is 15?
\item Out of 15,000 voters in a constituency, 60\% voted. Find the percentage of voters who did not vote. Can you now find how many actually did not vote?
\item  Meeta saves \rupee 4000 from her salary. If this is 10\% of her salary. What is her salary?
\item  A local cricket team played 20
	matches in one season. It won 25\% of them. How many matches did they win?
\item Reena’s mother said, to make idlis, you must take two parts rice and one part urad dal. What percentage of such a mixture would be rice and what percentage would be urad dal?
\item If \rupee 250 is to be divided amongst Ravi, Raju and Roy, so that Ravi gets two parts, Raju three parts and Roy five parts. How much money will each get? What will it be in percentages?
\item 	Divide 15 sweets between Manu and Sonu so that they get 20 \% and 80 \% of them respectively.
\item If angles of a triangle are in the ratio 2 : 3 : 4. Find the value of each angle.
\item A school team won 6 games this year against 4 games won last year. What is the per cent increase?
\item The number of illiterate persons in a country decreased from 150 lakhs to 100 lakhs in 10 years. What is the percentage of decrease?
\item Find Percentage of increase or decrease
	\begin{enumerate}
	\item  Price of shirt decreased from \rupee 280 to \rupee 210.
	\item Marks in a test increased from 20 to 30.
	\end{enumerate}
\item  My mother says, in her childhood petrol was \rupee 1 a litre. It is \rupee 52 per litre today. By what Percentage has the price gone up?
\item The cost of a flower vase is \rupee 120. If the shopkeeper sells it at a loss of 10\%, find the price at which it is sold.
\item Selling price of a toy car is \rupee 540. If the profit made by shopkeeper is 20\%, what is the cost price of this toy?
\item A shopkeeper bought a chair for \rupee 375 and sold it for \rupee 400. Find the gain Percentage. 
\item  Cost of an item is \rupee 50. It was sold with a profit of 12\%. Find the selling price. 
\item  An article was sold for \rupee 250 with a profit of 5\%. What was its cost price? 
\item  An item was sold for \rupee 540 at a loss of 5\%. What was its cost price?
\item Anita takes a loan of \rupee 5,000 at 15\% per year as rate of interest. Find the interest she has to pay at the end of one year.
\item \rupee 10,000 is invested at 5\% interest rate p.a. Find the interest at the end of one year.
\item  \rupee 3,500 is given at 7\% p.a. rate of interest. Find the interest which will be received at the end of two years.
\item  \rupee 6,050 is borrowed at 6.5\% rate of interest p.a.. Find the interest and the amount to be paid at the end of 3 years.
\item  \rupee 7,000 is borrowed at 3.5\% rate of interest p.a. borrowed for 2 years. Find the amount to be paid at the end of the second year.
\item If Manohar pays an interest of \rupee 750 for 2 years on a sum of \rupee 4,500, find the rate of interest.
\item You have \rupee 2,400 in your account and the interest rate is 5\%. After how many years would you earn \rupee 240 as interest.
\item On a certain sum the interest paid after 3 years is \rupee 450 at 5\% rate of interest per annum. Find the sum.
\item Tell what is the profit or loss in the following transactions. Also find profit per cent or loss per cent in each case. 
	\begin{enumerate}
\item  Gardening shears bought for \rupee 250 and sold for \rupee 325. 
\item  A refrigerater bought for \rupee 12,000 and sold at \rupee 13,500. 
\item  A cupboard bought for \rupee 2,500 and sold at \rupee 3,000. 
\item  A skirt bought for \rupee 250 and sold at \rupee 150.
	\end{enumerate}
\item Convert each part of the ratio to percentage
	\begin{enumerate}
		\item  3 : 1
		\item  2 : 3 : 5 
		\item  1:4 
		\item  1 : 2 : 5
	\end{enumerate}
\item  The population of a city decreased from 25,000 to 24,500. Find the percentage decrease.
\item  Arun bought a car for \rupee 3,50,000. The next year, the price went upto \rupee 3,70,000. What was the Percentage of price increase?
\item  I buy a T.V. for \rupee 10,000 and sell it at a profit of 20\%. How much money do I get for it?
\item  Juhi sells a washing machine for \rupee 13,500. She loses 20\% in the bargain. What was the price at which she bought it?
\item 
	\begin{enumerate}
		\item Chalk contains calcium, carbon and oxygen in the ratio 10:3:12. Find the percentage of carbon in chalk.
		\item  If in a stick of chalk, carbon is 3g, what is the weight of the chalk stick?
	\end{enumerate}
\item Aamani buys a book for \rupee 275 and sells it at a loss of 15\%. How much does she sell it for?
\item  Find the amount to be paid at the end of 3 years in each case
	\begin{enumerate}
\item  Principal = \rupee 1,200 at 12\% p.a.
\item  Principal = \rupee 7,500 at 5\% p.a.
	\end{enumerate}
\item  What rate gives \rupee 280 as interest on a sum of \rupee 56,000 in 2 years? 
\item  If Meena gives an interest of \rupee 45 for one year at 9\% rate p.a.. What is the sum she has borrowed?
\item	Find the missing values
	in
\tabref{tab:pgm}
	\begin{table}[H]
  \centering
  \input{tables/deci/pgm.tex}
  \caption{}
  \label{tab:pgm}
\end{table}
\item	Find the missing values
	in
\tabref{tab:tri}
	\begin{table}[H]
  \centering
  \input{tables/deci/tri.tex}
  \caption{}
  \label{tab:tri}
\end{table}
\end{enumerate}

\section{Programming}
\subsection{Formulae}
In a quiz, team A scored $a_1 = -40, a_2=10, a_3=0$ and team B scored $b_1=10, b_2=0, b_3=-40$ in three successive rounds.
\begin{enumerate}[label=\thesubsection.\arabic*, ref=\thesubsection.\theenumi]
\item  If the total scores are 
	\begin{align}
		a &= a_1+a_2+a_3
		\\
		b &= b_1+b_2+b_3
	\end{align}
	which team scored more? 
	\\
	\solution 
	\lstinputlisting{codes/prog/ifelse.c}
\item Write a function to compare the final scores.  Check for the cases when $a = -40, b = -40; a = 30, b = 20; a = -20, b = -10$.
	\\
	\solution 
	\lstinputlisting{codes/prog/func.c}
\item Use arrays and a for loop to evaluate 
	\begin{align}
		a &= \sum_{i=0}^{2}a_i
		\\
		b &= \sum_{i=0}^{2}b_i
	\end{align}
	\\
	\solution 
	\lstinputlisting{codes/prog/loop.c}
\item Revise the above code using only functions.
	\\
	\solution 
	\lstinputlisting{codes/prog/loopfunc.c}
\item Use files for the input data.
	\\
	\solution 
	\lstinputlisting{codes/prog/fileprogs/files.c}
\item Revise the files program using pointer arrays
	\\
	\solution 
	\lstinputlisting{codes/prog/fileprogs/pointer.c}
\item Revise the files program using only functions
	\\
	\solution 
	\lstinputlisting{codes/prog/fileprogs/filesfunc.c}
\end{enumerate}
Reduce to standard form
\begin{enumerate}[label=\thesubsection.\arabic*, ref=\thesubsection.\theenumi,resume*,itemsep=1ex]
	\item $\frac{-45}{30}$
		\\
		\solution
	\lstinputlisting{codes/prog/hcf.c}
\end{enumerate}
Use recursion for the following
\begin{enumerate}[label=\thesubsection.\arabic*, ref=\thesubsection.\theenumi,resume*]
		\item $9^3$
			\\
			\solution For integer powers, the exponent can be computed as
			\begin{align}
					x(n) = ax(n-1), x(0) = 1 
			\end{align}
			which results in the following C code.
	\lstinputlisting{codes/prog/recur.c}
\end{enumerate}
Use matrices for the following
\begin{enumerate}[label=\thesubsection.\arabic*, ref=\thesubsection.\theenumi,resume*]
	\item The difference in the measures of two complementary angles is 12\degree. Find the measures of the angles.
		\\
		\solution Let the angles be $x$ and $y$.  Then we have the following equations
		\begin{align}
			x+y = 180
			\\
			x-y = 12
		\end{align}
		which can be expressed as the matrix equation
		\begin{align}
		\myvec{1 & 1 \\ 1 & -1}\myvec{x\\y}= \myvec{180 \\ 12}
		\\
		\text{or, } \vec{A}\vec{x} = \vec{b}
		\end{align}
		The solution can be obtained as
		\begin{align}
			\vec{x} = \frac{\vec{A}^{\top}\vec{b}}{2}
		\end{align}
		using the following code
	\lstinputlisting{codes/prog/mat/matsol.c}
	by keeping the functions in .h files as below.  
	\lstinputlisting{codes/prog/mat/libs/matfun.h}
	\lstinputlisting{codes/prog/mat/libs/geofun.h}
\end{enumerate}

\subsection{NCERT}
Use ifelse  for the following
\begin{enumerate}[label=\thesubsection.\arabic*, ref=\thesubsection.\theenumi]
	\item $25 \times \brak{-21} = \brak{-21}\times 25 $
	\\
	\solution 
	\lstinputlisting{codes/prog/compeq.c}
	\begin{multicols}{2}
	\item $\brak{-48}\div\brak{8} = 48 \div \brak{-8}$
	\item $\brak{-23}\times  20 = 23 \times \brak{-20}$
	\item $90 \div \brak{-45} = \brak{-90}\div 45$
	\item $ \brak{-136}\div 4=136 \div \brak{-4} $
	\item $10\times  \sbrak{6+\brak{-2}} =10\times 6+10 \times \brak{-2} $
	\item $10\times  \sbrak{6-\brak{-2}} =10\times 6-10 \times \brak{-2} $
\end{multicols}
	\item $\brak{-15}\times  \sbrak{\brak{-7}-\brak{-1}} =\brak{-15}\times\brak{-7}-\brak{-15} \times \brak{-1} $
	\item $18\times  \sbrak{\brak{7}+\brak{-3}} =18\times\brak{7}+18 \times \brak{-3} $
	\item $\brak{-21}\times  \sbrak{\brak{-4}+\brak{-6}} =\brak{-21}\times\brak{-4}+\brak{-21} \times \brak{-6} $
	\item $\brak{-15}\times  \sbrak{\brak{-7}+\brak{-1}} =\brak{-15}\times\brak{-7}+\brak{-15} \times \brak{-1} $
\end{enumerate}
\begin{enumerate}[label=\thesubsection.\arabic*, ref=\thesubsection.\theenumi,resume*]
	\item 		An angle is greater than 45\degree. Is its complementary angle greater than 45\degree or equal to 45\degree or less than 45\degree?
\end{enumerate}
	Is it possible to have a triangle with the following sides? 
\begin{enumerate}[label=\thesubsection.\arabic*, ref=\thesubsection.\theenumi,resume*]
		\begin{multicols}{2}
\item $2 cm, 3 cm, 5 cm $ 
\item $3 cm, 6 cm, 7 cm $
\item $ 6 cm, 3 cm, 2 cm$
\item $10.2 cm, 5.8 cm, 4.5 cm$?
\end{multicols}
	\end{enumerate}
	 Between which two numbers can length of the third side fall?
	The lengths of two sides of a triangle are 
\begin{enumerate}[label=\thesubsection.\arabic*, ref=\thesubsection.\theenumi,resume*]
	\begin{multicols}{2}
		\item $6 cm$ and $8 cm$.
		\item $12 cm$ and $15 cm$.
		\end{multicols}
	\end{enumerate}
Which is greater?
\begin{enumerate}[label=\thesubsection.\arabic*, ref=\thesubsection.\theenumi,resume*,itemsep=1ex]
	\begin{multicols}{2}
	\item $\frac{1}{2}$ of $\frac{3}{4}$
	or $\frac{3}{5}$ of $\frac{5}{8}$
	\item $\frac{1}{2}$ of $\frac{6}{7}$
	or $\frac{2}{3}$ of $\frac{3}{7}$
\end{multicols}
\end{enumerate}
Identify which of the following pairs of angles are complementary and which are supplementary.
\begin{enumerate}[label=\thesubsection.\arabic*, ref=\thesubsection.\theenumi,resume*]
	\begin{multicols}{4}
	\item  65\degree, 115\degree
	\item  63\degree, 27\degree 
	\item  130\degree, 50\degree 
	\item 45\degree, 45\degree 
	\item  112\degree, 68\degree 
	\item  80\degree, 10\degree
\end{multicols}
\end{enumerate}
Which of these are negative rational numbers?
\begin{enumerate}[label=\thesubsection.\arabic*, ref=\thesubsection.\theenumi,resume*,itemsep=1ex]
	\begin{multicols}{4}
	\item $\frac{-2}{3}$
	\item $\frac{5}{7}$
	\item $\frac{3}{-5}$
	\item $\frac{6}{11}$
	\item $\frac{-2}{-9}$
	\item 0
\end{multicols}
\end{enumerate}
Compare the following and fill in the blanks
\begin{enumerate}[label=\thesubsection.\arabic*, ref=\thesubsection.\theenumi,resume*,itemsep=1ex]
	\begin{multicols}{4}
		\item $\frac{-5}{7}  \dots  \frac{2}{3}$
		\item $\frac{-8}{5}  \dots  \frac{-7}{4}$
		\item ${0}  \dots  \frac{-7}{6}$
		\item $\frac{-4}{5}  \dots  \frac{-5}{7}$
		\item $\frac{1}{-3}  \dots  \frac{-1}{4}$
		\item $\frac{-7}{8}  \dots  \frac{14}{-16}$
		\item $\frac{5}{-11}  \dots  \frac{-5}{11}$
		\item $\frac{2}{3}  \dots  \frac{5}{7}$
		\item $\frac{-3}{8}  \dots  \frac{-2}{7}$
		\item $\frac{-4}{3}  \dots  \frac{-3}{2}$
		\item $\frac{-7}{21}  \dots  \frac{3}{9}$
		\item $\frac{-3}{5}  \dots  \frac{-12}{20}$
		\item $\frac{-5}{-9}  \dots  \frac{5}{-9}$
		\item $\frac{-16}{20}  \dots  \frac{20}{-25}$
		\item $\frac{8}{-5}  \dots  \frac{-24}{15}$
		\item $\frac{-2}{-3}  \dots  \frac{2}{3}$
		\item $\frac{1}{3}  \dots  \frac{-1}{9}$
		\item $\frac{4}{-9}  \dots  \frac{-16}{36}$
		\item $\frac{2}{3}  \dots  \frac{5}{2}$
		\item $\frac{-1}{4}  \dots  \frac{1}{4}$
		\item $\frac{-5}{6}  \dots  \frac{-4}{3}$
		\item $-3\frac{2}{7}  \dots  -3\frac{4}{5}$
		\item $\frac{-3}{4}  \dots  \frac{2}{-3}$
\end{multicols}
\end{enumerate}
		Write four more numbers in the following pattern
\begin{enumerate}[label=\thesubsection.\arabic*, ref=\thesubsection.\theenumi,resume*,itemsep=1ex]
	\item $\frac{-1}{3},\frac{-2}{6},\frac{-3}{9},\frac{-4}{12}$
		\\
		\solution
	\lstinputlisting{codes/prog/seq.c}
	\begin{multicols}{3}
	\item $\frac{-3}{5},\frac{-6}{10},\frac{-9}{15},\frac{-12}{20}$
	\item $\frac{-1}{4},\frac{-2}{8},\frac{-3}{12}$
	\item $\frac{-1}{6},\frac{2}{-12},\frac{3}{-18},\frac{4}{-24}$
	\item $\frac{-2}{3},\frac{2}{-3},\frac{4}{-6},\frac{6}{-9}$
\end{multicols}
		\end{enumerate}
Reduce to standard form
\begin{enumerate}[label=\thesubsection.\arabic*, ref=\thesubsection.\theenumi,resume*,itemsep=1ex]
	\begin{multicols}{4}
	\item $\frac{36}{-24}$
	\item $\frac{-3}{-15}$
	\item $\frac{-18}{45}$
	\item $\frac{-12}{18}$
	\item $\frac{-8}{6}$
	\item $\frac{25}{45}$
	\item $\frac{-44}{72}$
	\item $\frac{-8}{10}$
\end{multicols}
\end{enumerate}
Use recursion for the following
\begin{enumerate}[label=\thesubsection.\arabic*, ref=\thesubsection.\theenumi,resume*]
	\item $\brak{-1}\times\brak{-2} \times \brak{-3} \times \brak{-4}$ 
		\\
		\solution In this case,
			\begin{align}
					x(n) = -nx(n-1), x(0) = 1 
			\end{align}
			Complete the code using the above equation.
	\begin{multicols}{4}
		\item $2^6$   
		\item $11^2$
		\item $5^4$
		\item $\brak{6^2}^{4}$
		\item $\brak{2^2}^{100}$
		\item $\brak{7^{50}}^{2}$
		\item $\brak{5^3}^{7}$
		\item ${2}^{5}\times {2}^{3}$
		\item ${4}^{3}\times {4}^{2}$
		\item ${5}^{3}\times {5}^{7}\times {5}^{12}$
		\item ${2}^{8}\div{2}^{3}$
		\item ${9}^{11}\div{9}^{7}$
		\item ${7}^{13}\div{7}^{10}$
		\item ${10}^{8}\div{10}^{4}$
		\item ${20}^{15}\div{20}^{13}$
		\item $\brak{-4}^{3}$
		\item $\brak{\frac{3}{5}}^{4}$
		\item $\brak{\frac{-4}{7}}^{5}$
		\item $\brak{-4}^{100}\times \brak{-4}^{20}$
		\end{multicols}
\end{enumerate}
Use arrays for the following
\begin{enumerate}[label=\thesubsection.\arabic*, ref=\thesubsection.\theenumi,resume*]
	\item $\brak{-12}\times  \brak{-11}\times \brak{10} $
		\\
		\solution The product can be expressed as
		\begin{align}
			y = \prod_{k=0}^{2}a_k
		\end{align}
		where
		\begin{align}
			\vec{a} = \myvec{-12 \\ -11 \\ 10}
		\end{align}
	The following code implements this
	\lstinputlisting{codes/prog/prodvec.c}
	\begin{multicols}{3}
	\item $\brak{9}\times\brak{-3} \times \brak{-6}$ 
	\item $\brak{-18}\times\brak{-5} \times \brak{-4}$ 
	\item $\brak{-3}\times\brak{-6} \times \brak{-2} \times \brak{-1}$ 
\end{multicols}
\end{enumerate}
Use matrices for the following
\begin{enumerate}[label=\thesubsection.\arabic*, ref=\thesubsection.\theenumi,resume*]
\item Among two supplementary angles the measure of the larger angle is $44\degree$  more than the measure of the smaller. Find their measures.
\end{enumerate}
\begin{enumerate}[label=\thesubsection.\arabic*, ref=\thesubsection.\theenumi,resume*]
\item 
	A collection of 10 chips with different colours is given in 
\figref{fig:percent2}.
Extend the table using a C program
to include a column for the percentage of chips of each colour.
\begin{figure}[H]
  \centering
  \begin{subfigure}{0.25\textwidth}
%    \includegraphics[width=\textwidth]{figs/percent2.jpg}
    %\documentclass{article}
%\usepackage{tikz}
%\usepackage[table]{xcolor}
%\usepackage[margin=1in]{geometry}
%
%\definecolor{lightblue}{RGB}{204, 238, 255}
%
%\begin{document}

\begin{center}
\begin{tikzpicture}
% Background
\fill[lightblue] (0,0) rectangle (4,3);

% Top row: G G G G (fully filled)
\foreach \i in {0,1,2,3} {
  \node at (\i + 0.5, 2.5) {\tikz \draw (0,0) circle (0.4cm) node {\textbf{G}};};
}

% Middle row: centered B B B (columns 0.5, 1.5, 2.5)
\foreach \i in {0,1,2} {
  \node at (\i + 0.5 + 0.5, 1.5) {\tikz \draw (0,0) circle (0.4cm) node {\textbf{B}};};
}

% Bottom row: centered R R R (columns 0.5, 1.5, 2.5)
\foreach \i in {0,1,2} {
  \node at (\i + 0.5 + 0.5, 0.5) {\tikz \draw (0,0) circle (0.4cm) node {\textbf{R}};};
}

\end{tikzpicture}
\end{center}

%\end{document}


    \caption{Chips}
  \end{subfigure}
  \hfill
\begin{subfigure}{0.4\textwidth}
%    \includegraphics[width=\textwidth]{figs/percent3.jpg}
	  \input{tables/deci/percent.tex}
    \caption{Table}
  \end{subfigure}
  \caption{}
  \label{fig:percent2}
\end{figure}
	\solution 
	\lstinputlisting{codes/prog/percent.c}
\end{enumerate}

\section{Data Handling}
\subsection{Formulae}
\begin{enumerate}[label=\thesubsection.\arabic*, ref=\thesubsection.\theenumi]
\item A batsman scored the following number of runs in 6 innings.  
	$$36, 35, 50, 46, 60, 55$$
	Calculate the mean runs scored by him in an inning.
	\\
	\solution
	\lstinputlisting{codes/data/avg.c}
\item The ages in years of 10 teachers of a school are
	$$32, 41, 28, 54, 35, 26, 23, 33, 38, 40$$
	\begin{enumerate}
		\item What is the age of the oldest teacher and that of the youngest teacher?
			\\
			\solution
	\lstinputlisting{codes/data/minmax.c}
		\item What is the range of the ages of the teachers?
		\item What is the mean age of these teachers?
	\end{enumerate}
\item Organize the following marks in a class assessment, in tabular form with columns as marks and frequency.
	\begin{gather}
	4, 6, 7, 5, 3, 5, 4, 5, 2, 6, 2, 5, 1, 9, 6, 5, 8, 4, 6, 7
	\end{gather}
	\begin{enumerate}
		\item Which number is the highest?
		\item Which number is the lowest?
		\item What is the range of the data?
		\item Find the arithmetic mean.
	\end{enumerate}
	\solution
	\lstinputlisting{codes/data/markfreq.c}
	\item 
Find the median of
		the group of 17 students with the following heights (in cm)
\begin{gather*}
106, 110, 123, 125, 117, 120, 112, 115, 
\\
110, 120, 115, 102, 115, 115, 109, 115, 101
\end{gather*}
	\\
	\solution
\lstinputlisting{codes/data/find-median.c}
\item	Sale of English and Hindi books in the years 1995, 1996, 1997 and 1998 are given below in
  \tabref{tab:sale}.  
	\begin{table}[H]
  \centering
  \input{tables/sale.tex}
  \caption{}
  \label{tab:sale}
\end{table}
Draw a double bar graph and answer the following questions
\begin{enumerate}
\item In which year was the difference in the sale of the two language books least?.
\item  Can you say that the demand for English books rose faster? Justify.
\end{enumerate}
	\solution
	\lstinputlisting{codes/data/bar.c}
\item Following are the margins of victory in the football matches of a league.  Find the mode of this data.
	\begin{gather*}
	1,3,2,5,1,4,6,2,5,2,2,2,4,1,2,3,1,1,2,3,2,6,4,3,2,
	\\
	1,1,4,2,1,5,3,3,2,3,2,4,2,1,2.
	\end{gather*}
	\solution
\lstinputlisting{codes/data/findmode.c}
\end{enumerate}

\subsection{NCERT}
\begin{enumerate}[label=\thesubsection.\arabic*, ref=\thesubsection.\theenumi]
\item Find the average of $4.2, 3.8$ and $7.6$.
\item Ashish studies for 4 hours, 5 hours and and 3 hours respectively on three consecutive days.  How many hours does he study daily on an average?
\item A cricketer scores the following runs in eight innings.
	$$58, 76, 40, 35, 46, 45, 0, 100$$
	Find the mean score.
\item Generate 
  \tabref{tab:player}
using a C program
	\begin{table}[H]
  \centering
  \input{tables/player.tex}
  \caption{}
  \label{tab:player}
\end{table}
and answer the following questions.
\begin{enumerate}
	\item Find the mean to determine $A's$ average number of points scored per game.
	\item Who is the best performer?
\end{enumerate}
\item The marks out of 100 obtained by a group of students in a science test are 85, 76, 90, 85, 39, 48, 56, 95, 81 and 75.  Find the 
	\begin{enumerate}
		\item Highest and lowest marks obtained by the students.
		\item Range of marks obtained.
		\item  Mean marks obtained by the group.
	\end{enumerate}
\item The enrolment in a school during six consecutive years was as follows  
	$$1555, 1670, 1750, 2013, 2540, 2820$$
	Find the mean enrolment of the school for this period.
\item The rainfall (in mm) in a city on 7 days a week was recorded as in 
  \tabref{tab:rainfall}.  Generate this table using a C program.
	\begin{table}[H]
  \centering
  \input{tables/rainfall.tex}
  \caption{}
  \label{tab:rainfall}
\end{table}
\item Find the range of the rainfall in the given data.
\item Find the mean rainfall for the week.
\item On how many days was the rainfall less than the mean rainfall 
\item The height of 10 girls was measured in cm and result was as follows
	$$135, 150, 139, 128, 151, 132, 146, 149, 143, 141.$$
	\begin{enumerate}
		\item What is the height of the tallest girl?
		\item What is the height of the shortest girl?
		\item What is the range of the data?
		\item What is the mean height of the girls?
		\item How many girls have heights more than the mean height?
	\end{enumerate}
\item To find out the weekly demand for different sizes of shirt, a shopkeeper kept records of sales of sizes as shown in 
\tabref{tab:shirt}.  This is the record for a week.  Find the mode of the data.
	\begin{table}[H]
  \centering
  \input{tables/shirt.tex}
  \caption{}
  \label{tab:shirt}
\end{table}
\item Find the mode of the given set of numbers
	$$1,1,1,2,2,2,2,3,4,4$$
\end{enumerate}
Find the mode of
\begin{enumerate}[label=\thesubsection.\arabic*, ref=\thesubsection.\theenumi,resume*]
\item 	
		\begin{gather*}
		2,6,5,3,0,4,3,2,4,5,2,4
	\end{gather*}
\item 
		\begin{gather*}
		2,4,16,12,14,14,16,14,10,14,18,14
	\end{gather*}
	\item 
		\begin{gather*}
		2,2,2,3,3,4,5,5,5,6,6,8
	\end{gather*}
	\item 
		\begin{gather*}
		12, 14, 12, 16, 15, 13, 14, 18, 19, 12, 14, 15, 16, 15, 16, 16,
	\\
		15, 17, 13, 16, 16, 15, 15, 13, 15, 17, 15, 14, 15, 13, 15, 14
		\end{gather*}
	\item Heights (in cm) of 25 children given below
		\begin{gather*}
		168, 165, 163, 160, 163, 161, 162, 164, 163, 162,
164, 163,	160, 
		\\
	 163, 160, 165, 163, 162, 163, 164, 163, 160, 165, 163, 162 
	\end{gather*}
		What is the mode of their heights? What do we understand by mode here?
\end{enumerate}
\begin{enumerate}[label=\thesubsection.\arabic*, ref=\thesubsection.\theenumi,resume*]
\item Your friend found the median and the mode of a given data. Describe and correct your friends error if any
\begin{gather*}
	35, 32, 35, 42, 38, 32, 34 
\end{gather*}
Median = 42, Mode = 32.
\item Find the median of the data: 24, 36, 46, 17, 18, 25, 35.
\item The scores in mathematics test (out of 25) of 15 students is as follows
	\begin{gather*}
	19, 25, 23, 20, 9, 20, 15, 10, 5, 16, 25, 20, 24, 12, 20 
\end{gather*}
Find the mode and median of this data. Are they same?
\item The runs scored in a cricket match by 11 players is as follows 
	\begin{gather*}
	6, 15, 120, 50, 100, 80, 10, 15, 8, 10, 15
\end{gather*}
Find the mean, mode and median of this data. Are the three same?
\item The weights (in kg.) of 15 students of a class are
	\begin{gather*}
	38, 42, 35, 37, 45, 50, 32, 43, 43, 40, 36, 38, 43, 38, 47
\end{gather*}
\begin{enumerate}
	\item  Find the mode and median of this data. 
	\item Is there more than one mode?
\end{enumerate}
\item Find the mode and median of the data
	\begin{gather*}
	13, 16, 12, 14, 19, 12, 14, 13, 14
\end{gather*}
\item
	The data 
	\begin{gather*}
	6, 4, 3, 8, 9, 12, 13, 9 
\end{gather*}
has mean 9.  True or False?
\item Two hundred students of 6th and 7th classes were asked to name their favourite colour so as to decide upon what should be the colour of their school building. The results are shown in the following table
  in   \tabref{tab:colour}.
	 Represent the given data on a bar graph
	\begin{table}[H]
  \centering
  \input{tables/colour.tex}
  \caption{}
  \label{tab:colour}
\end{table}
Answer the following questions with the help of the bar graph
	\begin{enumerate}
	\item Which is the most preferred colour and which is the least preferred? 
	\item How many colours are there in all? What are they?
	\end{enumerate}
\item Following data 
  in \tabref{tab:marks}
	gives total marks (out of 600) obtained by six children of a particular class. Represent the data on a bar graph.
	\begin{table}[H]
  \centering
  \input{tables/marks.tex}
  \caption{}
  \label{tab:marks}
\end{table}
\item Consider the following two collections of data 
in
  \tabref{tab:hours}
	giving the average daily hours of sunshine in two cities Aberdeen and Margate for all the twelve months of the year. These cities are near the south pole and hence have only a few hours of sunshine each day.
	In a particular month, which city has more sunshine hours?  Explain through a double bar graph.
	\begin{table}[H]
  \centering
  \input{tables/hours.tex}
  \caption{}
  \label{tab:hours}
\end{table}
\item A mathematics teacher wants to see, whether the new technique of teaching she applied after quarterly test was effective or not. She takes the scores of the 5 weakest children in the quarterly test (out of 25) and in the half yearly test (out of 25) which are listed in 
  \tabref{tab:students}.  Is her technique effective?
	\begin{table}[H]
  \centering
  \input{tables/students.tex}
  \caption{}
  \label{tab:students}
\end{table}
\item 
	Number of children in six different classes are given below
  in \tabref{tab:children}.  
 Represent the data on a bar graph.
	\begin{table}[H]
  \centering
  \input{tables/children.tex}
  \caption{}
  \label{tab:children}
\end{table}
\begin{enumerate}
	\item How would you choose a scale? 
	\item  Which class has the maximum number of children? And the minimum? 
	\item  Find the ratio of students of class sixth to the students of class eight.
\end{enumerate}
\item The performance of a student in 1st Term and 2nd Term is given
in \tabref{tab:term}.
	Draw a double bar graph choosing appropriate scale and answer the following
	\begin{table}[H]
  \centering
  \input{tables/term.tex}
  \caption{}
  \label{tab:term}
\end{table}
\begin{enumerate}
\item In which subject, has the child improved his performance the most? 
\item In which subject is the improvement the least? 
\item Has the performance gone down in any subject?
\end{enumerate}
\item Consider this data 
  in \tabref{tab:sport}
	collected from a survey of a colony.
	\begin{table}[H]
  \centering
  \input{tables/sport.tex}
  \caption{}
  \label{tab:sport}
\end{table}
\begin{enumerate}
\item Draw a double bar graph choosing an appropriate scale. What do you infer from the bar graph?
\item Which sport is most popular? 
\item Which is more preferred, watching or participating in sports?
\end{enumerate}
\end{enumerate}
Write the following rational numbers in ascending order
\begin{enumerate}[label=\thesubsection.\arabic*, ref=\thesubsection.\theenumi,resume*,itemsep=1ex]
	\begin{multicols}{3}
	\item $\frac{-3}{5},\frac{-2}{5},\frac{-1}{5}$
	\item $\frac{-1}{3},\frac{-2}{9},\frac{-4}{3}$
	\item $\frac{-3}{7},\frac{-3}{2},\frac{-3}{4}$
	\end{multicols}
\end{enumerate}
\begin{enumerate}[label=\thesubsection.\arabic*, ref=\thesubsection.\theenumi,resume*]
\item Find
	\begin{enumerate}
	\begin{multicols}{3}
	\item $2.7\times 4$
	\item $1.8\times 1.2$
	\item $2.3\times 4.35$
	\end{multicols}
	\end{enumerate}
and arrange the products in descending order.
\end{enumerate}

\section{Math Library}
\begin{enumerate}[label=\thesection.\arabic*, ref=\thesection.\theenumi]
\item $\triangle ABC$ is right-angled at $C$. If $AC = 5 cm$ and $BC = 12 cm$ find the length of $AB$.
	\\
	\solution
	\lstinputlisting{codes/math/baudh.c}
\item Determine whether the triangle whose lengths of sides are $3 cm, 4 cm, 5 cm$ is a right-angled triangle.
\item $PQR$ is a triangle, right-angled at $P$. If $PQ = 10cm$ and $PR = 24 cm$, find $QR$.
\item $ABC$ is a triangle, right-angled at $C$. If $AB = 25 cm$ and $AC = 7 cm$, find $BC$.
\item A $15 m$ long ladder reached a window $12 m$ high from the ground on placing it against a wall at a distance $a$. Find the distance of the foot of the ladder from the wall.
\item  Which of the following can be the sides of a right triangle? 
\begin{enumerate}
	\item $2.5 cm,6.5 cm, 6 cm.$ 
	\item $ 2 cm, 2 cm, 5 cm.$ 
	\item $ 1.5 cm, 2cm, 2.5 cm.$
\end{enumerate}
\item A tree is broken at a height of $5 m$ from the ground and its top touches the ground at a distance of $12 m$ from the base of the tree. Find the original height of the tree.
\item Find the perimeter of the rectangle whose length is $40 cm$ and a diagonal is $41 cm$. 
\item The diagonals of a rhombus measure $16 cm$ and $30 cm$. Find its perimeter.
\item Find the values of the following expressions for $x = 2$. 
	\begin{enumerate}
\item  $19-5x^2$
\item  $100 – 10x^3$
	\end{enumerate}
\item Find the value of the following expressions when $n = – 2.$ 
	\begin{enumerate}
\item  $5n^2 + 5n – 2 $
\item  $n^3 + 5n^2 + 5n – 2$
	\end{enumerate}
\item Find the value of the following expressions for $a = 3, b = 2$. 
	\begin{enumerate}
\item $ a^2+2ab+b^2$
\item $ a^3 – b^3$
\end{enumerate}
\item If $m = 2$, find the value of
	\begin{enumerate}
\item $ 3m^2 – 2m – 7 $
\item $ \frac{5m^4}{ 2}$
\end{enumerate}
\item  If $p = – 2$, find the value of
	\begin{enumerate}
\item  $– 3p^2 + 4p + 7 $
\item  $– 2p^3 – 3p^2 + 4p + 7$
\end{enumerate}
\item  Find the value of the following expressions, when $x = –1$ 
	\begin{enumerate}
\item $ 2x^2 – x – 2$
\item $ x^2 + 2x +1$
\end{enumerate}
\item  If $a = 2, b = – 2$, find the value of
	\begin{enumerate}
\item  $ a^2 + b^2$
\item  $a^2 + ab + b^2 $
\item  $a^2 – b^2$
\end{enumerate}
\item  When $a = 0, b = – 1$, find the value of the given expressions
	\begin{enumerate}
\item $2a^2 + b^2 + 1 $
\item $2a^2b + 2ab^2 + ab $
\item $a^2 + ab + 2$
\end{enumerate}
\item If $z = 10$, find the value of $z^3 – 3(z – 10)$. 
\item  If $p = – 10$, find the value of $p^2 – 2p – 100$
\item  Simplify the expression and find its value when $a = 5$ and $b = – 3$
	$$ 2a^2 + ab + 3 $$
\item What is the circumference of a circle of diameter 10 cm?
	\\
	\solution
	\lstinputlisting{codes/math/circ.c}
\item What is the circumference of a circular disc of radius 14 cm?
\item The radius of a circular pipe is 10 cm. What length of a tape is required to wrap once around the pipe?
\item Sudhanshu divides a circular disc of radius 7 cm in two equal parts. What is the perimeter of each semicircular shape disc?
\item Find the area of a circle of radius 30 cm?
\item Diameter of a circular garden is 9.8 m. Find its area.
\item Find the circumference of the circles with the following radius  
	\begin{enumerate}
		\item 28 mm
\item  14 cm
\end{enumerate}
\item  Find the area of the following circles, given that the radius is
	\begin{enumerate}
\item 14 mm 
\item 5 cm
\item 21 cm 
\item  diameter = 49 m
\end{enumerate}
\item If the circumference of a circular sheet is 154 m, find its radius. Also find the area of the sheet. 
\item A gardener wants to fence a circular garden of diameter 21m. Find the length of the rope he needs to purchase, if he makes 2 rounds of fence. Also find the cost of the rope, if it costs \rupee 4 per meter. 
\item  From a circular sheet of radius 4 cm, a circle of radius 3 cm is removed. Find the area of the remaining sheet. 
\item Seema wants to put a lace on the edge of a circular table cover of diameter 1.5 m. Find the length of the lace required and also find its cost if one meter of the lace costs
\rupee 15. 
\item Find the cost of polishing a circular table-top of diameter 1.6 m, if the rate of polishing is \rupee $15/m^2$. 
\item Shalya took a wire of length 44 cm and bent it into the shape of a circle. Find the radius of that circle. Also find its area. If the same wire is bent into the shape of a square, what will be the length of each of its sides? Which figure encloses more
area, the circle or the square? 
\item From a circular card sheet of radius 14 cm, two circles of radius 3.5 cm and a rectangle of length 3 cm and breadth 1cm are removed. 
 Find the area of the remaining sheet. 
\item A circle of radius 2 cm is cut out from a square piece of an aluminium sheet of side 6 cm. What is the area of the left over aluminium sheet? 
\item  The circumference of a circle is 31.4 cm. Find the radius and the area of the circle. 
\item A circular flower bed is surrounded by a path 4 m wide. The diameter of the flower bed is 66 m. What is the area of this path? 
\item A circular flower garden has an area of $314 m^2$. A sprinkler at the centre of the garden can cover an area that has a radius of 12 m. Will the sprinkler water the entire garden? 
\item How many times a wheel of radius 28 cm must rotate to go 352 m? 
\item The minute hand of a circular clock is 15 cm.
	How far does the tip of the minute hand move in 1 hour? 
\item The two sides of the parallelogram ABCD are 6 cm and 4 cm. The height corresponding to the base CD is 3 cm. Find the
	\begin{enumerate}
\item 	area of the parallelogram. 
\item the height corresponding to the base AD.
\end{enumerate}
Find the value of
\begin{enumerate}[label=\thesection.\arabic*, ref=\thesection.\theenumi,resume*,itemsep=1ex]
	\begin{multicols}{4}
		\item ${2}\times {10}^{3}$
		\item ${7}^{2}\times {2}^{2}$
		\item ${4}^{3}\times {2}^{3}$
		\item ${5}^{6}\times \brak{-2}^{6}$
		\item $\brak{-2}^{4}\times {-3}^{4}$
		\item ${2}^{3}\times {5}$
		\item ${3}^{}\times {4}^{4}$
		\item ${5}^{2}\times {3}^{3}$
		\item ${2}^{4}\times {3}^{2}$
		\item ${3}^{2}\times {10}^{4}$
		\item $\brak{-3}\times {-2}^{3}$
		\item $\brak{-3}^{2}\times {-5}^{2}$
		\item ${-2}^{3}\times {-10}^{3}$
		\item ${4}^{5}\div{3}^{5}$
		\item ${5}^{6}\div \brak{-2}^{6}$
		\item $\brak{\frac{3^7}{3^2}}\times 3^{5}$
		\item ${2}^{3}\times {2}^{2}\times {5}^{5}$
		\item ${6}^{2}\times {6}^{4}\div {6}^{3}$
		\item ${8}^{2}\div {2}^{3}$
		\item $\brak{2^2}^{3}\times {3}^{6}\times {5}^{6}$
	\item $\frac{\brak{12}^{4}\times {9}^{3}\times {4}}{\brak{6}^{3}\times {8}^{2}\times {27}}$
	\item $\frac{\brak{2}\times {3}^{4}\times {2}^{5}}{{9}\times {4}^{2}}$
		\item ${3}^{2}\times {3}^{4}\times {3}^{8}$
		\item ${6}^{15}\div {6}^{10}$
		\item $\brak{5^2}^{3}\div  {5}^{3}$
		\item ${2}^{5}\times {5}^{5}$
		\item $\brak{3^4}^{3}$
		\item $\brak{{2}^{20}\div{2}^{15}}\times{2}^{3}$
		\item $\frac{\brak{2}^{3}\times {3}^{4}\times {4}^{}}{{3}\times {32}^{}}$
		\item $\brak{{5^2}^{3}\times {5^2}^{}}\div  {5}^{7}$
		\item ${25}^{4}\div  {5}^{3}$
	\item $\frac{\brak{3}\times {7}^{2}\times {11}^{8}}{{21}\times {11}^{3}}$
	\item $\frac{\brak{3}^7}{{3}^{4}\times {3}^{3}}$
		\item $\brak{{2}^{3}\times{2}}^{2}$
		\item $\frac{\brak{2^5}^{2}\times {7}^{3}}{{8}^3\times {7}}$
	\item $\frac{\brak{3}^{5}\times {10}^{5}\times {25}}{\brak{5}^{7}\times {6}^{5}}$
	\end{multicols}
\end{enumerate}
Find the logarithms
\begin{enumerate}[label=\thesection.\arabic*, ref=\thesection.\theenumi,resume*]
	\item 512 base 2
	\\
	\solution
	\lstinputlisting{codes/math/ln.c}
	\begin{multicols}{4}
	\item 256 base 2
	\item  343 base 7
	\item  729 base 3
	\item  3125 base 5
	\end{multicols}
\end{enumerate}
Identify the greater number, wherever possible, in each of the following
\begin{enumerate}[label=\thesection.\arabic*, ref=\thesection.\theenumi,resume*]
	\begin{multicols}{2}
\item $4^3      \text{ or } 3^4$               
\item $ 5^3     \text{ or } 3^5$ 
\item $ 100^2   \text{ or } 2^{100} $ 
\item $2^{10}  \text{ or } 10^2$
\item $2^{3}  \text{ or } 3^{2}$
\item $8^{2}  \text{ or } 2^{8}$
\item $2.7\times 10^{12}  \text{ or } 1.5\times 10^8$
\item $4\times 10^{14}  \text{ or } 3\times 10^{17}$
\end{multicols}
\end{enumerate}
Express each of the following as product of powers of their prime factors
\begin{enumerate}[label=\thesection.\arabic*, ref=\thesection.\theenumi,resume*]
	\begin{multicols}{4}
\item 	648
\item 	405
\item 	540
\item 	3600
\item 	72
\item 	432
\item 	1000
\item 	16000
\item 	270
\item 	768
\item $108 \times 192$
\item $629\times 65$
\end{multicols}
\end{enumerate}
Use ifelse  for the following
\begin{enumerate}[label=\thesection.\arabic*, ref=\thesection.\theenumi,resume*]
\item $10\times 10^{11} = 100^{11}$
\item $2^{3} > 5^{2}$
\item $2^3\times 3^{2} = 6^{5}$
\item $3^0 = 1000^{0}$
\end{enumerate}
\end{enumerate}

\section{Random Numbers}
\begin{enumerate}[label=\thesection.\arabic*, ref=\thesection.\theenumi]
	\item Take a board marked from -104 to 104 as shown in 
	\figref{fig:game1}.
	\item Take a bag containing two blue and two red dice.  Number of dots on the blue dice indicate positive integers and number of dots on the red dice indicate negative integers.
	\item Every player will place his/her counter at zero.
	\item Each player will take out two dice at a time from the bag and throw them.
	\item After every throw, the player has to multiply the numbers marked on the dice.
	\item If the product is a positive integer then the player will move his counter towards 104; if the product is a negative integer then the player will move his counter towards -104.
	\item The player who reaches either -104 or 104 first is the winner.
		\begin{figure}[H]
  \centering
  \includegraphics[width=\columnwidth]{figs/game1.jpg}
  \caption{}
  \label{fig:game1}
\end{figure}
\end{enumerate}
%
\begin{enumerate}[label=\thesection.\arabic*, ref=\thesection.\theenumi,resume*]
	\item Write a program to simulate the game.  Give the inputs manually.
	\\
	\solution 
	\lstinputlisting{codes/rv/game.c}
	\item Revise the program by replacing the second player with the computer.  The computer generates the inputs randomly as follows 
		\begin{enumerate}
			\item Generate the numbers on all the dice using a uniform distribution ranging from 1 to 6.
			\item Simulate the blue and red dice through a Bernoulli distribution having values 1 and -1.
		\end{enumerate}
	\solution 
	\lstinputlisting{codes/rv/rgame.c}
\item Now revise the program so that both players are simulated by the computer.
\end{enumerate}

\newpage
\appendices
\section{Triangle}
%\numberwithin{equation}{section}
Consider a triangle with vertices
		\begin{align}
			\label{eq:app-tri-pts}
			\vec{A} = \myvec{1 \\ -1},\,
			\vec{B} = \myvec{-4 \\ 6},\,
			\vec{C} = \myvec{-3 \\ -5}
		\end{align}
\subsection{Sides}
\input{triangle/vector}
\subsection{Median}
\begin{enumerate}[label=\thesubsection.\arabic*.,ref=\thesubsection.\theenumi]
\numberwithin{equation}{enumi}
\item If $\vec{D}$ divides $BC$ in the ratio $k : 1$,
		\begin{align}
			\vec{D}= \frac{k\vec{C}+\vec{B}}{k+1}
	  \label{eq:app-section_formula}
		\end{align}
		Find the mid points $\vec{D}, \vec{E}, \vec{F}$ of the sides $BC, CA$ and $AB$ respectively.
	\\
		\documentclass[journal]{IEEEtran}
\usepackage[a5paper, margin=10mm]{geometry}
%\usepackage{lmodern} % Ensure lmodern is loaded for pdflatex
\usepackage{tfrupee} % Include tfrupee package


\setlength{\headheight}{1cm} % Set the height of the header box
\setlength{\headsep}{0mm}     % Set the distance between the header box and the top of the text


%\usepackage[a5paper, top=10mm, bottom=10mm, left=10mm, right=10mm]{geometry}

%
\usepackage{gvv-book}
\usepackage{gvv}
%\setlength{\intextsep}{10pt} % Space between text and floats

\makeindex

\begin{document}
\bibliographystyle{IEEEtran}
\onecolumn


\title{
	%\begin{flushleft}
	\begin{center}
	%MATRICES \\ In Geometry
	C Programming in Middle School
%	Progressions
	\\
\rule{0.4\columnwidth}{0.4pt}
%\end{flushleft}
\end{center}
}
\author{
\vspace{11cm}
	%\begin{flushleft}
	\begin{center}
\includegraphics[width=0.2\columnwidth]{figs/logo.jpg}
\\
		{\huge	G. V. V. Sharma}\\Associate Professor,\\Department of Electrical Engineering, \\ IIT Hyderabad
	\end{center}
	%\end{flushleft}
%\IEEEpubid{\makebox[\columnwidth]{978-1-7281-5966-1/20/\$31.00 ©2020 IEEE \hfill} \hspace{\columnsep}\makebox[\columnwidth]{ }}
}
\maketitle

\newpage
\section*{About this Book}

This book introduces C programming for middle school children based on the
 NCERT mathematics textbook of Class 7.  

There is no copyright, so readers are free to print and share.  

This book is dedicated to my Hindi teacher in middle school, Shri Mandavi.
\begin{flushright}
\today
\end{flushright}
Github: https://github.com/gadepall/cprog
		\\
License: https://creativecommons.org/licenses/by-sa/3.0/
\\
and
\\
https://www.gnu.org/licenses/fdl-1.3.en.html

\newpage


\tableofcontents

\newpage
%\twocolumn
\onecolumn


%\renewcommand{\theequation}{\theenumi}
\numberwithin{equation}{enumi}
%\numberwithin{figure}{enumi}
%\numberwithin{figure}{section}
%\numberwithin{figure}{subsection}
\renewcommand{\thefigure}{\theenumi}
\renewcommand{\thetable}{\theenumi}

\section{Integers}
\subsection{Formulae}
\begin{enumerate}[label=\thesubsection.\arabic*, ref=\thesubsection.\theenumi]
\item Do the following addition through a C program
	$$17+23$$
	\\
	\solution
	\lstinputlisting{codes/integer/add.c}
\item Do the following subtraction through a C program
$$7-9$$
	\\
	\solution
	\lstinputlisting{codes/integer/sub.c}
\item Mulitply the following through a C program
	$$4\times \brak{-8}$$
	\\
	\solution
	\lstinputlisting{codes/integer/mul.c}
\item Perform the following division
	$$\brak{-100}\div 5$$
	\\
	\solution
	\lstinputlisting{codes/integer/div.c}
\end{enumerate}

\subsection{NCERT}
Compute the following
\begin{enumerate}[label=\thesubsection.\arabic*, ref=\thesubsection.\theenumi]
	\begin{multicols}{4}
\item $\brak{-75}+18$
\item $19+\brak{-25}$
\item $27+\brak{-27}$
\item $\brak{-20}+0$
\item $\brak{-35}+\brak{-10}$
\item $\brak{-10}+3$
\item $17-(-21)$
\item $8\times(-2)$
\item $3\times(-7)$
\item $10\times(-1)$
\item $6\times(-19)$
\item $12\times(-32)$
\item $7\times(-22)$
\item $15\times(-16)$
\item $21\times(-32)$
\item $(-42)\times 12$
\item $(-55)\times 15$
\item $(-5)\times \brak{-6}$
\item $\brak{-6}\times(-7)$
\item $3\times(-1)$
\item $\brak{-1}\times 225$
\item $\brak{-21}\times(-30)$
\item $\brak{-316}\times(-1)$
\item	$\brak{-81}\div 9$
\item	$\brak{-75}\div 5$
\item	$\brak{-32}\div 2$
\item	$125\div \brak{-25}$
\item	$80\div \brak{-5}$
\item	$64\div \brak{-16}$
\item	$\brak{-30}\div 10$
\item	$50\div \brak{-5}$
\item	$\brak{-36}\div \brak{-9}$
\item	$\brak{-49}\div \brak{-49}$
\end{multicols}
\end{enumerate}
\begin{enumerate}[label=\thesubsection.\arabic*, ref=\thesubsection.\theenumi,resume*]
	\begin{multicols}{2}
\item	$13\div \sbrak{\brak{-2}+1}$
\item	$\brak{-31}\div \sbrak{\brak{-30}+\brak{-1}}$
\item	$\sbrak{\brak{-36}\div 12}\div \brak{3}$
\item	$\sbrak{\brak{-6}+5}\div \sbrak{\brak{-2}+1}$
	\end{multicols}
\end{enumerate}
Fill in the blanks
\begin{enumerate}[label=\thesubsection.\arabic*, ref=\thesubsection.\theenumi,resume*]
	\begin{multicols}{2}
		\item	$20 \div \rule{1cm}{0.1pt}=-2$
		\item	$\rule{1cm}{0.1pt}\div 4=-3$
	\end{multicols}
\end{enumerate}
Find the values of the following expressions for $x = 2$. 
	\begin{multicols}{3}
\begin{enumerate}[label=\thesubsection.\arabic*, ref=\thesubsection.\theenumi,resume*]
\item $x + 4$
\item  $4x – 3$ 
\item $ x – 2$
\item $ 3x – 5 $
\item $ 9 – 5x $
\item $ x + 7 + 4 (x – 5)$
\item $ 3 (x + 2) + 5x – 7 $
\item $ 6x + 5 (x – 2) $
\item $ 4(2x – 1) + 3x + 11$
\end{enumerate}
\end{multicols}
If $x = – 2$, find the value of
	\begin{multicols}{2}
\begin{enumerate}[label=\thesubsection.\arabic*, ref=\thesubsection.\theenumi,resume*]
\item $5x – 2$
\item $4p + 7$
\end{enumerate}
\end{multicols}
Find the value of the following expressions for $a = 3, b = 2$. 
	\begin{multicols}{2}
\begin{enumerate}[label=\thesubsection.\arabic*, ref=\thesubsection.\theenumi,resume*]
\item $a + b$				
\item $ 7a – 4b $
\end{enumerate}
\end{multicols}
Find the value of the following expressions, when $x = –1$ 
\begin{enumerate}[label=\thesubsection.\arabic*, ref=\thesubsection.\theenumi,resume*]
\item $ 2x – 7$
\item $ – x + 2$ 
\end{enumerate}
When $a = 0, b = – 1$, find the value of the given expressions
\begin{enumerate}[label=\thesubsection.\arabic*, ref=\thesubsection.\theenumi,resume*]
\item $2a + 2b$
\end{enumerate}
Simplify these expressions and find their values if $x = 3, a = – 1, b = – 2$. 
	\begin{multicols}{4}
\begin{enumerate}[label=\thesubsection.\arabic*, ref=\thesubsection.\theenumi,resume*]
\item $ 2x +4 $
\item $6 - 4x$
\item $6 – 5a $
\item $6 – 8b $
\item $3a – 2b-9 $
\end{enumerate}
\end{multicols}
\begin{enumerate}[label=\thesubsection.\arabic*, ref=\thesubsection.\theenumi,resume*]
	\item In a test (+5) marks are given for every correct answer and (-2) marks for every incorrect answer.  
		\begin{enumerate}
			\item Radhika answered all the questions and scored 30 marks though she got 10 correct answers.
			\item Jay also answered all the questions and scored (-12) marks though he got 4 correct answers.  How many incorrect answers had they  attempted?
		\end{enumerate}
	\item A shopkeeper earns a profit of \rupee 1 by selling one pen and incurs a loss of 40 paise per pencil while selling pencils of her old stock.  
		\begin{enumerate}
			\item In a particular month she incurs a loss of \rupee 5.  In this period she sold 45 pens.  How many pencils did she sell in this period?
			\item In the next month she earns neither profit nor loss.  If she sold 70 pens, how many pencils did she sell?
		\end{enumerate}
			\item The temperature at 12 noon was 10\degree C above zero. If it decreases at the rate of 2\degree C per hour unitl midnight, at what time would the temperature be 8\degree C below zero? 
	\item In a class test (+3) marks are given for every correct answer and (-2) marks for every incorrect answer and no marks for not attempting any question.   
		\begin{enumerate}
			\item Radhika scored 20 marks.  If she got 12 correct answers, how many questions has she attempted incorrectly?
			\item Mohini scores -5 marks in this test, though she has got 7 correct answers.   How many questions has she attempted incorrectly?
		\end{enumerate}
	\item An elevator descends a mine shaft at the rate of $6m/min$.  If the descent starts from $10m$ above the ground, how long will it take to reach $-350m$.
	\item What is the measure of the complement of each of the following angles? 
		\begin{enumerate}
			\begin{multicols}{4}
	\item 45\degree
\item 65\degree 
\item 41\degree 
\item 54\degree 
			\end{multicols}
\end{enumerate}
\item What will be the measure of the supplement of each one of the following angles? 
		\begin{enumerate}
			\begin{multicols}{4}
\item 100\degree
\item  90\degree 
\item  55\degree 
\item  125\degree
			\end{multicols}
\end{enumerate}
\item An exterior angle of a triangle is of measure $70\degree$ and one of its interior opposite angles is of measure 25\degree. Find the measure of the other interior opposite angle.
\item The two interior opposite angles of an exterior angle of a triangle are $60\degree$ and 80\degree. Find the measure of the exterior angle.
\item 	Two angles of a triangle are $30\degree$ and 80\degree. Find the third angle. 
\item  One of the angles of a triangle is $80\degree$ and the other two angles are equal. Find the measure of each of the equal angles.
\item  The three angles of a triangle are in the ratio 1:2:1. Find all the angles of the triangle. Classify the triangle in two different ways.
\item One of the sides and the corresponding height of a parallelogram are 4 cm and 3 cm respectively. Find the area of the parallelogram.
\item Find the height $x$ if the area of the parallelogram is 24 $cm^2$ and the base is 4 cm.
\item Find BC, if the area of the triangle ABC is 36 $cm^2$ and the height AD is 3 cm.
		\end{enumerate}

\section{Decimal Numbers}
\subsection{Formulae}
Find
\begin{enumerate}[label=\thesubsection.\arabic*,ref=\thesubsection.\theenumi,itemsep=1ex]
	\item $\frac{5}{3}+\frac{3}{5}$
		\\
		\solution
	\lstinputlisting{codes/deci/add.c}
	\item $\frac{2}{3}-\frac{3}{7}$
		\\
		\solution
	\lstinputlisting{codes/deci/sub.c}
	\item $5.6\times 1.4$
		\\
		\solution
	\lstinputlisting{codes/deci/mul.c}
	\item $ 37.8\div 1.4$
		\\
		\solution
	\lstinputlisting{codes/deci/div.c}
\end{enumerate}

\subsection{NCERT}
Find
\begin{enumerate}[label=\thesubsection.\arabic*,ref=\thesubsection.\theenumi,itemsep=1ex]
	\begin{multicols}{4}
%
	\item $\frac{2}{7}\times 3$
	\item $\frac{9}{7}\times 6$
	\item $\frac{1}{8}\times 3$
	\item $\frac{13}{11}\times 6$
	\item $\frac{2}{5}\times 2$
	\item $3\times 5\frac{1}{5}$
	\item $5\times 6\frac{3}{4}$
	\item $7\times 2\frac{1}{4}$
	\item $4\times 6\frac{1}{3}$
	\item $6\times 3\frac{1}{4}$
	\item $8\times 3\frac{2}{5}$
	\item $\frac{1}{2}\times \frac{1}{7}$
	\item $\frac{1}{5}\times \frac{1}{7}$
	\item $\frac{1}{3}\times \frac{4}{5}$
	\item $\frac{2}{3}\times \frac{1}{5}$
	\item $\frac{8}{3}\times \frac{4}{7}$
	\item $\frac{3}{4}\times \frac{2}{3}$
	\item $\frac{2}{3}\times 2\frac{2}{3}$
	\item $\frac{2}{7}\times \frac{7}{9}$
	\item $\frac{3}{8}\times \frac{6}{4}$
	\item $\frac{9}{5}\times \frac{3}{5}$
	\item $\frac{1}{3}\times \frac{15}{8}$
	\item $\frac{11}{2}\times \frac{3}{10}$
	\item $\frac{4}{5}\times \frac{12}{7}$
	\item $\frac{2}{5}\times 5\frac{1}{4}$
	\item $6\frac{2}{5}\times \frac{7}{9}$
	\item $\frac{3}{2}\times 5\frac{1}{3}$
	\item $\frac{5}{6}\times 2\frac{3}{7}$
	\item $3\frac{2}{5}\times \frac{4}{7}$
	\item $2\frac{3}{5}\times 3$
	\item $3\frac{4}{7}\times \frac{3}{5}$
	\item $\frac{2}{3}\times \rule{0.5cm}{0.1pt}=\frac{10}{30}$
	\item $\frac{3}{5}\times \rule{0.5cm}{0.1pt}=\frac{24}{75}$
	\item $7 \div \frac{2}{5}$
	\item $6 \div \frac{4}{7}$
	\item $2 \div \frac{8}{9}$
	\item $\frac{3}{5} \div \frac{1}{2}$
	\item $\frac{1}{2} \div \frac{3}{5}$
	\item $2\frac{1}{2} \div \frac{3}{5}$
	\item $5\frac{1}{6} \div \frac{9}{2}$
	\item $12 \div \frac{3}{4}$
	\item $14 \div \frac{5}{6}$
	\item $8 \div \frac{7}{3}$
	\item $4 \div \frac{8}{3}$
	\item $3 \div 2\frac{1}{3}$
	\item $5 \div 3\frac{4}{7}$
	\item $\frac{7}{3}\div 2$
	\item $\frac{4}{9}\div 5$ 
	\item $\frac{6}{13}\div 7$
	\item $4\frac{1}{3}\div 3$
	\item $3\frac{1}{2}\div 4$
	\item $4\frac{3}{7}\div 7$
	\item $\frac{2}{5} \div \frac{1}{2}$
	\item $\frac{4}{9} \div \frac{2}{3}$
	\item $\frac{3}{7} \div \frac{8}{7}$
	\item $2\frac{1}{3} \div \frac{3}{5}$
	\item $3\frac{1}{2} \div \frac{8}{3}$
	\item $\frac{2}{5} \div 1\frac{1}{2}$
	\item $3\frac{1}{5} \div 1\frac{2}{3}$
	\item $2\frac{1}{5} \div 1\frac{1}{5}$
	\item $0.2\times 6$
	\item $8\times 4.6$
	\item $2.71\times 5$
	\item $20.1\times 4$
	\item $0.05\times 7$
	\item $211.02\times 4$
	\item $2\times 0.86$
	\item $2.5\times 0.3$
	\item $0.1\times 51.7$
	\item $0.2\times 316.8$
	\item $1.3\times 3.1$
	\item $0.5\times 0.05$
	\item $11.2\times 0.15$
	\item $1.07\times 0.02$
	\item $10.05\times 1.05$
	\item $101.01\times 0.01$
	\item $100.01\times 1.1$
	\item $7.75\times 0.25$
	\item $42.8\times 0.02$
	\item $0.4 \div  2$
	\item $0.35 \div  5$
	\item $2.48 \div  4$
	\item $65.4 \div  6$
	\item $651.2 \div  4$
	\item $14.49 \div  7$
	\item $3.96  \div  4$
	\item $0.80 \div  5$
	\item $7 \div 3.5$
	\item $ 36\div 0.2$
	\item $3.25 \div 0.5$
	\item $ 30.94\div 0.7$
	\item $ 0.5\div 0.25$
	\item $ 7.75\div 0.25$
	\item $ 76.5\div 0.15$
	\item $ 2.73\div 1.3$
	\item $\frac{5}{4}+\frac{-11}{4}$
	\item $\frac{-9}{10}+\frac{22}{15}$
	\item $\frac{-3}{-11}+\frac{5}{9}$
	\item $\frac{-8}{19}+\frac{-2}{57}$
	\item $\frac{-2}{3}+{0}$
	\item $\frac{-13}{7}+\frac{6}{7}$
	\item $\frac{19}{5}+\frac{-7}{5}$
	\item $\frac{-5}{6}+\frac{-3}{11}$
	\item $-2\frac{1}{3}+4\frac{3}{5}$
	\item $\frac{7}{24}-\frac{17}{36}$
	\item $\frac{5}{63}-\frac{-6}{21}$
	\item $\frac{-6}{13}-\frac{-7}{15}$
	\item $\frac{-3}{8}-\frac{7}{11}$
	\item $-2\frac{1}{9}-{6}$
	\item $\frac{7}{9}-\frac{2}{5}$
	\item $2\frac{1}{5}-\frac{-1}{3}$
	\item $\frac{9}{2}\times \frac{-7}{4}$
	\item $\frac{3}{10}\times {-9}$
	\item $\frac{-6}{5}\times \frac{9}{11}$
	\item $\frac{3}{7}\times \frac{-2}{5}$
	\item $\frac{3}{11}\times \frac{2}{5}$
	\item $\frac{3}{-5}\times \frac{-5}{3}$
	\item ${-4}\div \frac{2}{3}$
	\item $\frac{-3}{5}\div {2}$
	\item $\frac{-4}{5}\div {-3}$
	\item $\frac{-1}{8}\div \frac{3}{4}$
	\item $\frac{-2}{13}\div \frac{1}{7}$
	\item $\frac{-7}{12}\div \frac{-2}{13}$
	\item $\frac{3}{13}\div \frac{-4}{65}$
	\item $\frac{-3}{5}\times 7\frac{}{}$
	\item $\frac{-6}{5}\times -2\frac{}{}$
	\item $\frac{-3}{4}\times \frac{1}{7}$
	\item $\frac{2}{3}\times \frac{-5}{9}$
	\item $\frac{2}{3}\times \frac{-7}{8}$
	\item $\frac{-6}{7}\times \frac{5}{7}$
	\end{multicols}
\end{enumerate}
Find
\begin{enumerate}[label=\thesubsection.\arabic*,ref=\thesubsection.\theenumi,resume*,itemsep=1ex]
	\begin{multicols}{2}
	\item $\frac{1}{2}$ of
		\begin{enumerate}
			\item $2\frac{3}{4}$
			\item $4\frac{2}{9}$
		\end{enumerate}
	\item $\frac{5}{8}$ of
		\begin{enumerate}
			\item $3\frac{5}{6}$
			\item $9\frac{2}{3}$
		\end{enumerate}
	\end{multicols}
\end{enumerate}
Find
\begin{enumerate}[label=\thesubsection.\arabic*,ref=\thesubsection.\theenumi,resume*]
	\begin{multicols}{2}
	\item $\frac{1}{4}$ of
		\begin{enumerate}
			\item $\frac{1}{4}$
			\item $\frac{3}{5}$
			\item $\frac{4}{3}$
		\end{enumerate}
	\item $\frac{1}{7}$ of
		\begin{enumerate}
			\item $\frac{2}{9}$
			\item $\frac{6}{5}$
			\item $\frac{3}{10}$
		\end{enumerate}
	\end{multicols}
\end{enumerate}
\begin{enumerate}[label=\thesubsection.\arabic*,ref=\thesubsection.\theenumi,resume*]
	\item  In a class of 40 students $\frac{1}{5}$ of the total number of students like to study English, 
$\frac{2}{5}$ of the total number like to study Mathematics and the remaining students like to study Science.
\begin{enumerate}
	\item How many students like to study English?
	\item How many students like to study Mathematics?
	\item How many students like to study Science?
\end{enumerate}
\item Vidya and Pratap went for a picnic.  Their mother gave them a water bottle that contained 5 litres of water.  Vidya consumed $\frac{2}{5}$ of the water.  Pratap consumed the remaining water.
	\begin{enumerate}
		\item How much water did Vidya drink?
		\item What fraction of the total quantity of water did Pratap drink?
	\end{enumerate}
\item Shaili plants 4 saplings in a row, in her garden.  The distance between two adjacent saplings is $\frac{3}{4}m$. Find the distance between the first and the last sapling.
\item Lipika reads a book for $1\frac{3}{4}$ hours everyday.  She reads the entire book in 6 days.  How many hours in all were required by her to read the book.
\item A car runs $16km$ using 1 litre of petrol.  How much distance will it cover using $2\frac{3}{4}$ litres of petrol.
\item The side of an equilateral triangle is $3.5cm$.  Find its perimeter.  
\item The length of a rectangle is $7.1cm$ and its breadth is $2.5cm$.  What is its area?
\item Find the area of a rectangle whose length is $5.7cm$ and breadth is $3cm$. 
\item A two wheeler covers a distance of $55.3km$ in one litre of petrol.  How much will it cover in 10 litres of petrol?
\item Savita was preparing a design to decorate her classroom.  She needed a few coloured strips of paper of length $1.9cm$ each.  She had a strip of coloured paper of length $9.5cm$.  How many pieces of the required length will she get out of this strip? 
\item Each side of a regular polygon is $2.5cm$ in length.  The perimeter of the polygon is $12.5cm$.  How many sides does the polygon have?
\item A car covers a distance of $89.1km$ in $2.2$ hours.  What is the average distance covered by it in 1 hour?
\item A vehicle covers a distance of $43.2km$ in in $2.4$ litres of petrol.  How much will it cover in one litre of petrol?
\item 	Mala has a collection of bangles. She has 20 gold bangles and 10 silver bangles. What is the percentage of bangles of each type? Can you put it in the tabular form?
\item 	Out of 25 children in a class, 15 are girls. What is the percentage of girls?
\item Out of 32 students, 8 are absent. What per cent of the students are absent? 
\item There are 25 radios, 16 of them are out of order. What per cent of radios are out of order?
\item  A shop has 500 items, out of which 5 are defective. What per cent are defective? 
\item  There are 120 voters, 90 of them voted yes. What per cent voted yes?
\item 	A survey of 40 children showed that 25\% liked playing football. How many children liked playing football?
\item	Rahul bought a sweater and saved  \rupee 200 when a discount of 25\% was given. What was the price of the sweater before the discount?
\item 	9 is 25\% of what number? 
\item 	75\% of what number is 15?
\item Out of 15,000 voters in a constituency, 60\% voted. Find the percentage of voters who did not vote. Can you now find how many actually did not vote?
\item  Meeta saves \rupee 4000 from her salary. If this is 10\% of her salary. What is her salary?
\item  A local cricket team played 20
	matches in one season. It won 25\% of them. How many matches did they win?
\item Reena’s mother said, to make idlis, you must take two parts rice and one part urad dal. What percentage of such a mixture would be rice and what percentage would be urad dal?
\item If \rupee 250 is to be divided amongst Ravi, Raju and Roy, so that Ravi gets two parts, Raju three parts and Roy five parts. How much money will each get? What will it be in percentages?
\item 	Divide 15 sweets between Manu and Sonu so that they get 20 \% and 80 \% of them respectively.
\item If angles of a triangle are in the ratio 2 : 3 : 4. Find the value of each angle.
\item A school team won 6 games this year against 4 games won last year. What is the per cent increase?
\item The number of illiterate persons in a country decreased from 150 lakhs to 100 lakhs in 10 years. What is the percentage of decrease?
\item Find Percentage of increase or decrease
	\begin{enumerate}
	\item  Price of shirt decreased from \rupee 280 to \rupee 210.
	\item Marks in a test increased from 20 to 30.
	\end{enumerate}
\item  My mother says, in her childhood petrol was \rupee 1 a litre. It is \rupee 52 per litre today. By what Percentage has the price gone up?
\item The cost of a flower vase is \rupee 120. If the shopkeeper sells it at a loss of 10\%, find the price at which it is sold.
\item Selling price of a toy car is \rupee 540. If the profit made by shopkeeper is 20\%, what is the cost price of this toy?
\item A shopkeeper bought a chair for \rupee 375 and sold it for \rupee 400. Find the gain Percentage. 
\item  Cost of an item is \rupee 50. It was sold with a profit of 12\%. Find the selling price. 
\item  An article was sold for \rupee 250 with a profit of 5\%. What was its cost price? 
\item  An item was sold for \rupee 540 at a loss of 5\%. What was its cost price?
\item Anita takes a loan of \rupee 5,000 at 15\% per year as rate of interest. Find the interest she has to pay at the end of one year.
\item \rupee 10,000 is invested at 5\% interest rate p.a. Find the interest at the end of one year.
\item  \rupee 3,500 is given at 7\% p.a. rate of interest. Find the interest which will be received at the end of two years.
\item  \rupee 6,050 is borrowed at 6.5\% rate of interest p.a.. Find the interest and the amount to be paid at the end of 3 years.
\item  \rupee 7,000 is borrowed at 3.5\% rate of interest p.a. borrowed for 2 years. Find the amount to be paid at the end of the second year.
\item If Manohar pays an interest of \rupee 750 for 2 years on a sum of \rupee 4,500, find the rate of interest.
\item You have \rupee 2,400 in your account and the interest rate is 5\%. After how many years would you earn \rupee 240 as interest.
\item On a certain sum the interest paid after 3 years is \rupee 450 at 5\% rate of interest per annum. Find the sum.
\item Tell what is the profit or loss in the following transactions. Also find profit per cent or loss per cent in each case. 
	\begin{enumerate}
\item  Gardening shears bought for \rupee 250 and sold for \rupee 325. 
\item  A refrigerater bought for \rupee 12,000 and sold at \rupee 13,500. 
\item  A cupboard bought for \rupee 2,500 and sold at \rupee 3,000. 
\item  A skirt bought for \rupee 250 and sold at \rupee 150.
	\end{enumerate}
\item Convert each part of the ratio to percentage
	\begin{enumerate}
		\item  3 : 1
		\item  2 : 3 : 5 
		\item  1:4 
		\item  1 : 2 : 5
	\end{enumerate}
\item  The population of a city decreased from 25,000 to 24,500. Find the percentage decrease.
\item  Arun bought a car for \rupee 3,50,000. The next year, the price went upto \rupee 3,70,000. What was the Percentage of price increase?
\item  I buy a T.V. for \rupee 10,000 and sell it at a profit of 20\%. How much money do I get for it?
\item  Juhi sells a washing machine for \rupee 13,500. She loses 20\% in the bargain. What was the price at which she bought it?
\item 
	\begin{enumerate}
		\item Chalk contains calcium, carbon and oxygen in the ratio 10:3:12. Find the percentage of carbon in chalk.
		\item  If in a stick of chalk, carbon is 3g, what is the weight of the chalk stick?
	\end{enumerate}
\item Aamani buys a book for \rupee 275 and sells it at a loss of 15\%. How much does she sell it for?
\item  Find the amount to be paid at the end of 3 years in each case
	\begin{enumerate}
\item  Principal = \rupee 1,200 at 12\% p.a.
\item  Principal = \rupee 7,500 at 5\% p.a.
	\end{enumerate}
\item  What rate gives \rupee 280 as interest on a sum of \rupee 56,000 in 2 years? 
\item  If Meena gives an interest of \rupee 45 for one year at 9\% rate p.a.. What is the sum she has borrowed?
\item	Find the missing values
	in
\tabref{tab:pgm}
	\begin{table}[H]
  \centering
  \input{tables/deci/pgm.tex}
  \caption{}
  \label{tab:pgm}
\end{table}
\item	Find the missing values
	in
\tabref{tab:tri}
	\begin{table}[H]
  \centering
  \input{tables/deci/tri.tex}
  \caption{}
  \label{tab:tri}
\end{table}
\end{enumerate}

\section{Programming}
\subsection{Formulae}
In a quiz, team A scored $a_1 = -40, a_2=10, a_3=0$ and team B scored $b_1=10, b_2=0, b_3=-40$ in three successive rounds.
\begin{enumerate}[label=\thesubsection.\arabic*, ref=\thesubsection.\theenumi]
\item  If the total scores are 
	\begin{align}
		a &= a_1+a_2+a_3
		\\
		b &= b_1+b_2+b_3
	\end{align}
	which team scored more? 
	\\
	\solution 
	\lstinputlisting{codes/prog/ifelse.c}
\item Write a function to compare the final scores.  Check for the cases when $a = -40, b = -40; a = 30, b = 20; a = -20, b = -10$.
	\\
	\solution 
	\lstinputlisting{codes/prog/func.c}
\item Use arrays and a for loop to evaluate 
	\begin{align}
		a &= \sum_{i=0}^{2}a_i
		\\
		b &= \sum_{i=0}^{2}b_i
	\end{align}
	\\
	\solution 
	\lstinputlisting{codes/prog/loop.c}
\item Revise the above code using only functions.
	\\
	\solution 
	\lstinputlisting{codes/prog/loopfunc.c}
\item Use files for the input data.
	\\
	\solution 
	\lstinputlisting{codes/prog/fileprogs/files.c}
\item Revise the files program using pointer arrays
	\\
	\solution 
	\lstinputlisting{codes/prog/fileprogs/pointer.c}
\item Revise the files program using only functions
	\\
	\solution 
	\lstinputlisting{codes/prog/fileprogs/filesfunc.c}
\end{enumerate}
Reduce to standard form
\begin{enumerate}[label=\thesubsection.\arabic*, ref=\thesubsection.\theenumi,resume*,itemsep=1ex]
	\item $\frac{-45}{30}$
		\\
		\solution
	\lstinputlisting{codes/prog/hcf.c}
\end{enumerate}
Use recursion for the following
\begin{enumerate}[label=\thesubsection.\arabic*, ref=\thesubsection.\theenumi,resume*]
		\item $9^3$
			\\
			\solution For integer powers, the exponent can be computed as
			\begin{align}
					x(n) = ax(n-1), x(0) = 1 
			\end{align}
			which results in the following C code.
	\lstinputlisting{codes/prog/recur.c}
\end{enumerate}
Use matrices for the following
\begin{enumerate}[label=\thesubsection.\arabic*, ref=\thesubsection.\theenumi,resume*]
	\item The difference in the measures of two complementary angles is 12\degree. Find the measures of the angles.
		\\
		\solution Let the angles be $x$ and $y$.  Then we have the following equations
		\begin{align}
			x+y = 180
			\\
			x-y = 12
		\end{align}
		which can be expressed as the matrix equation
		\begin{align}
		\myvec{1 & 1 \\ 1 & -1}\myvec{x\\y}= \myvec{180 \\ 12}
		\\
		\text{or, } \vec{A}\vec{x} = \vec{b}
		\end{align}
		The solution can be obtained as
		\begin{align}
			\vec{x} = \frac{\vec{A}^{\top}\vec{b}}{2}
		\end{align}
		using the following code
	\lstinputlisting{codes/prog/mat/matsol.c}
	by keeping the functions in .h files as below.  
	\lstinputlisting{codes/prog/mat/libs/matfun.h}
	\lstinputlisting{codes/prog/mat/libs/geofun.h}
\end{enumerate}

\subsection{NCERT}
Use ifelse  for the following
\begin{enumerate}[label=\thesubsection.\arabic*, ref=\thesubsection.\theenumi]
	\item $25 \times \brak{-21} = \brak{-21}\times 25 $
	\\
	\solution 
	\lstinputlisting{codes/prog/compeq.c}
	\begin{multicols}{2}
	\item $\brak{-48}\div\brak{8} = 48 \div \brak{-8}$
	\item $\brak{-23}\times  20 = 23 \times \brak{-20}$
	\item $90 \div \brak{-45} = \brak{-90}\div 45$
	\item $ \brak{-136}\div 4=136 \div \brak{-4} $
	\item $10\times  \sbrak{6+\brak{-2}} =10\times 6+10 \times \brak{-2} $
	\item $10\times  \sbrak{6-\brak{-2}} =10\times 6-10 \times \brak{-2} $
\end{multicols}
	\item $\brak{-15}\times  \sbrak{\brak{-7}-\brak{-1}} =\brak{-15}\times\brak{-7}-\brak{-15} \times \brak{-1} $
	\item $18\times  \sbrak{\brak{7}+\brak{-3}} =18\times\brak{7}+18 \times \brak{-3} $
	\item $\brak{-21}\times  \sbrak{\brak{-4}+\brak{-6}} =\brak{-21}\times\brak{-4}+\brak{-21} \times \brak{-6} $
	\item $\brak{-15}\times  \sbrak{\brak{-7}+\brak{-1}} =\brak{-15}\times\brak{-7}+\brak{-15} \times \brak{-1} $
\end{enumerate}
\begin{enumerate}[label=\thesubsection.\arabic*, ref=\thesubsection.\theenumi,resume*]
	\item 		An angle is greater than 45\degree. Is its complementary angle greater than 45\degree or equal to 45\degree or less than 45\degree?
\end{enumerate}
	Is it possible to have a triangle with the following sides? 
\begin{enumerate}[label=\thesubsection.\arabic*, ref=\thesubsection.\theenumi,resume*]
		\begin{multicols}{2}
\item $2 cm, 3 cm, 5 cm $ 
\item $3 cm, 6 cm, 7 cm $
\item $ 6 cm, 3 cm, 2 cm$
\item $10.2 cm, 5.8 cm, 4.5 cm$?
\end{multicols}
	\end{enumerate}
	 Between which two numbers can length of the third side fall?
	The lengths of two sides of a triangle are 
\begin{enumerate}[label=\thesubsection.\arabic*, ref=\thesubsection.\theenumi,resume*]
	\begin{multicols}{2}
		\item $6 cm$ and $8 cm$.
		\item $12 cm$ and $15 cm$.
		\end{multicols}
	\end{enumerate}
Which is greater?
\begin{enumerate}[label=\thesubsection.\arabic*, ref=\thesubsection.\theenumi,resume*,itemsep=1ex]
	\begin{multicols}{2}
	\item $\frac{1}{2}$ of $\frac{3}{4}$
	or $\frac{3}{5}$ of $\frac{5}{8}$
	\item $\frac{1}{2}$ of $\frac{6}{7}$
	or $\frac{2}{3}$ of $\frac{3}{7}$
\end{multicols}
\end{enumerate}
Identify which of the following pairs of angles are complementary and which are supplementary.
\begin{enumerate}[label=\thesubsection.\arabic*, ref=\thesubsection.\theenumi,resume*]
	\begin{multicols}{4}
	\item  65\degree, 115\degree
	\item  63\degree, 27\degree 
	\item  130\degree, 50\degree 
	\item 45\degree, 45\degree 
	\item  112\degree, 68\degree 
	\item  80\degree, 10\degree
\end{multicols}
\end{enumerate}
Which of these are negative rational numbers?
\begin{enumerate}[label=\thesubsection.\arabic*, ref=\thesubsection.\theenumi,resume*,itemsep=1ex]
	\begin{multicols}{4}
	\item $\frac{-2}{3}$
	\item $\frac{5}{7}$
	\item $\frac{3}{-5}$
	\item $\frac{6}{11}$
	\item $\frac{-2}{-9}$
	\item 0
\end{multicols}
\end{enumerate}
Compare the following and fill in the blanks
\begin{enumerate}[label=\thesubsection.\arabic*, ref=\thesubsection.\theenumi,resume*,itemsep=1ex]
	\begin{multicols}{4}
		\item $\frac{-5}{7}  \dots  \frac{2}{3}$
		\item $\frac{-8}{5}  \dots  \frac{-7}{4}$
		\item ${0}  \dots  \frac{-7}{6}$
		\item $\frac{-4}{5}  \dots  \frac{-5}{7}$
		\item $\frac{1}{-3}  \dots  \frac{-1}{4}$
		\item $\frac{-7}{8}  \dots  \frac{14}{-16}$
		\item $\frac{5}{-11}  \dots  \frac{-5}{11}$
		\item $\frac{2}{3}  \dots  \frac{5}{7}$
		\item $\frac{-3}{8}  \dots  \frac{-2}{7}$
		\item $\frac{-4}{3}  \dots  \frac{-3}{2}$
		\item $\frac{-7}{21}  \dots  \frac{3}{9}$
		\item $\frac{-3}{5}  \dots  \frac{-12}{20}$
		\item $\frac{-5}{-9}  \dots  \frac{5}{-9}$
		\item $\frac{-16}{20}  \dots  \frac{20}{-25}$
		\item $\frac{8}{-5}  \dots  \frac{-24}{15}$
		\item $\frac{-2}{-3}  \dots  \frac{2}{3}$
		\item $\frac{1}{3}  \dots  \frac{-1}{9}$
		\item $\frac{4}{-9}  \dots  \frac{-16}{36}$
		\item $\frac{2}{3}  \dots  \frac{5}{2}$
		\item $\frac{-1}{4}  \dots  \frac{1}{4}$
		\item $\frac{-5}{6}  \dots  \frac{-4}{3}$
		\item $-3\frac{2}{7}  \dots  -3\frac{4}{5}$
		\item $\frac{-3}{4}  \dots  \frac{2}{-3}$
\end{multicols}
\end{enumerate}
		Write four more numbers in the following pattern
\begin{enumerate}[label=\thesubsection.\arabic*, ref=\thesubsection.\theenumi,resume*,itemsep=1ex]
	\item $\frac{-1}{3},\frac{-2}{6},\frac{-3}{9},\frac{-4}{12}$
		\\
		\solution
	\lstinputlisting{codes/prog/seq.c}
	\begin{multicols}{3}
	\item $\frac{-3}{5},\frac{-6}{10},\frac{-9}{15},\frac{-12}{20}$
	\item $\frac{-1}{4},\frac{-2}{8},\frac{-3}{12}$
	\item $\frac{-1}{6},\frac{2}{-12},\frac{3}{-18},\frac{4}{-24}$
	\item $\frac{-2}{3},\frac{2}{-3},\frac{4}{-6},\frac{6}{-9}$
\end{multicols}
		\end{enumerate}
Reduce to standard form
\begin{enumerate}[label=\thesubsection.\arabic*, ref=\thesubsection.\theenumi,resume*,itemsep=1ex]
	\begin{multicols}{4}
	\item $\frac{36}{-24}$
	\item $\frac{-3}{-15}$
	\item $\frac{-18}{45}$
	\item $\frac{-12}{18}$
	\item $\frac{-8}{6}$
	\item $\frac{25}{45}$
	\item $\frac{-44}{72}$
	\item $\frac{-8}{10}$
\end{multicols}
\end{enumerate}
Use recursion for the following
\begin{enumerate}[label=\thesubsection.\arabic*, ref=\thesubsection.\theenumi,resume*]
	\item $\brak{-1}\times\brak{-2} \times \brak{-3} \times \brak{-4}$ 
		\\
		\solution In this case,
			\begin{align}
					x(n) = -nx(n-1), x(0) = 1 
			\end{align}
			Complete the code using the above equation.
	\begin{multicols}{4}
		\item $2^6$   
		\item $11^2$
		\item $5^4$
		\item $\brak{6^2}^{4}$
		\item $\brak{2^2}^{100}$
		\item $\brak{7^{50}}^{2}$
		\item $\brak{5^3}^{7}$
		\item ${2}^{5}\times {2}^{3}$
		\item ${4}^{3}\times {4}^{2}$
		\item ${5}^{3}\times {5}^{7}\times {5}^{12}$
		\item ${2}^{8}\div{2}^{3}$
		\item ${9}^{11}\div{9}^{7}$
		\item ${7}^{13}\div{7}^{10}$
		\item ${10}^{8}\div{10}^{4}$
		\item ${20}^{15}\div{20}^{13}$
		\item $\brak{-4}^{3}$
		\item $\brak{\frac{3}{5}}^{4}$
		\item $\brak{\frac{-4}{7}}^{5}$
		\item $\brak{-4}^{100}\times \brak{-4}^{20}$
		\end{multicols}
\end{enumerate}
Use arrays for the following
\begin{enumerate}[label=\thesubsection.\arabic*, ref=\thesubsection.\theenumi,resume*]
	\item $\brak{-12}\times  \brak{-11}\times \brak{10} $
		\\
		\solution The product can be expressed as
		\begin{align}
			y = \prod_{k=0}^{2}a_k
		\end{align}
		where
		\begin{align}
			\vec{a} = \myvec{-12 \\ -11 \\ 10}
		\end{align}
	The following code implements this
	\lstinputlisting{codes/prog/prodvec.c}
	\begin{multicols}{3}
	\item $\brak{9}\times\brak{-3} \times \brak{-6}$ 
	\item $\brak{-18}\times\brak{-5} \times \brak{-4}$ 
	\item $\brak{-3}\times\brak{-6} \times \brak{-2} \times \brak{-1}$ 
\end{multicols}
\end{enumerate}
Use matrices for the following
\begin{enumerate}[label=\thesubsection.\arabic*, ref=\thesubsection.\theenumi,resume*]
\item Among two supplementary angles the measure of the larger angle is $44\degree$  more than the measure of the smaller. Find their measures.
\end{enumerate}
\begin{enumerate}[label=\thesubsection.\arabic*, ref=\thesubsection.\theenumi,resume*]
\item 
	A collection of 10 chips with different colours is given in 
\figref{fig:percent2}.
Extend the table using a C program
to include a column for the percentage of chips of each colour.
\begin{figure}[H]
  \centering
  \begin{subfigure}{0.25\textwidth}
%    \includegraphics[width=\textwidth]{figs/percent2.jpg}
    \input{figs/testfig.tex}
    \caption{Chips}
  \end{subfigure}
  \hfill
\begin{subfigure}{0.4\textwidth}
%    \includegraphics[width=\textwidth]{figs/percent3.jpg}
	  \input{tables/deci/percent.tex}
    \caption{Table}
  \end{subfigure}
  \caption{}
  \label{fig:percent2}
\end{figure}
	\solution 
	\lstinputlisting{codes/prog/percent.c}
\end{enumerate}

\section{Data Handling}
\subsection{Formulae}
\begin{enumerate}[label=\thesubsection.\arabic*, ref=\thesubsection.\theenumi]
\item A batsman scored the following number of runs in 6 innings.  
	$$36, 35, 50, 46, 60, 55$$
	Calculate the mean runs scored by him in an inning.
	\\
	\solution
	\lstinputlisting{codes/data/avg.c}
\item The ages in years of 10 teachers of a school are
	$$32, 41, 28, 54, 35, 26, 23, 33, 38, 40$$
	\begin{enumerate}
		\item What is the age of the oldest teacher and that of the youngest teacher?
			\\
			\solution
	\lstinputlisting{codes/data/minmax.c}
		\item What is the range of the ages of the teachers?
		\item What is the mean age of these teachers?
	\end{enumerate}
\item Organize the following marks in a class assessment, in tabular form with columns as marks and frequency.
	\begin{gather}
	4, 6, 7, 5, 3, 5, 4, 5, 2, 6, 2, 5, 1, 9, 6, 5, 8, 4, 6, 7
	\end{gather}
	\begin{enumerate}
		\item Which number is the highest?
		\item Which number is the lowest?
		\item What is the range of the data?
		\item Find the arithmetic mean.
	\end{enumerate}
	\solution
	\lstinputlisting{codes/data/markfreq.c}
	\item 
Find the median of
		the group of 17 students with the following heights (in cm)
\begin{gather*}
106, 110, 123, 125, 117, 120, 112, 115, 
\\
110, 120, 115, 102, 115, 115, 109, 115, 101
\end{gather*}
	\\
	\solution
\lstinputlisting{codes/data/find-median.c}
\item	Sale of English and Hindi books in the years 1995, 1996, 1997 and 1998 are given below in
  \tabref{tab:sale}.  
	\begin{table}[H]
  \centering
  \input{tables/sale.tex}
  \caption{}
  \label{tab:sale}
\end{table}
Draw a double bar graph and answer the following questions
\begin{enumerate}
\item In which year was the difference in the sale of the two language books least?.
\item  Can you say that the demand for English books rose faster? Justify.
\end{enumerate}
	\solution
	\lstinputlisting{codes/data/bar.c}
\item Following are the margins of victory in the football matches of a league.  Find the mode of this data.
	\begin{gather*}
	1,3,2,5,1,4,6,2,5,2,2,2,4,1,2,3,1,1,2,3,2,6,4,3,2,
	\\
	1,1,4,2,1,5,3,3,2,3,2,4,2,1,2.
	\end{gather*}
	\solution
\lstinputlisting{codes/data/findmode.c}
\end{enumerate}

\subsection{NCERT}
\begin{enumerate}[label=\thesubsection.\arabic*, ref=\thesubsection.\theenumi]
\item Find the average of $4.2, 3.8$ and $7.6$.
\item Ashish studies for 4 hours, 5 hours and and 3 hours respectively on three consecutive days.  How many hours does he study daily on an average?
\item A cricketer scores the following runs in eight innings.
	$$58, 76, 40, 35, 46, 45, 0, 100$$
	Find the mean score.
\item Generate 
  \tabref{tab:player}
using a C program
	\begin{table}[H]
  \centering
  \input{tables/player.tex}
  \caption{}
  \label{tab:player}
\end{table}
and answer the following questions.
\begin{enumerate}
	\item Find the mean to determine $A's$ average number of points scored per game.
	\item Who is the best performer?
\end{enumerate}
\item The marks out of 100 obtained by a group of students in a science test are 85, 76, 90, 85, 39, 48, 56, 95, 81 and 75.  Find the 
	\begin{enumerate}
		\item Highest and lowest marks obtained by the students.
		\item Range of marks obtained.
		\item  Mean marks obtained by the group.
	\end{enumerate}
\item The enrolment in a school during six consecutive years was as follows  
	$$1555, 1670, 1750, 2013, 2540, 2820$$
	Find the mean enrolment of the school for this period.
\item The rainfall (in mm) in a city on 7 days a week was recorded as in 
  \tabref{tab:rainfall}.  Generate this table using a C program.
	\begin{table}[H]
  \centering
  \input{tables/rainfall.tex}
  \caption{}
  \label{tab:rainfall}
\end{table}
\item Find the range of the rainfall in the given data.
\item Find the mean rainfall for the week.
\item On how many days was the rainfall less than the mean rainfall 
\item The height of 10 girls was measured in cm and result was as follows
	$$135, 150, 139, 128, 151, 132, 146, 149, 143, 141.$$
	\begin{enumerate}
		\item What is the height of the tallest girl?
		\item What is the height of the shortest girl?
		\item What is the range of the data?
		\item What is the mean height of the girls?
		\item How many girls have heights more than the mean height?
	\end{enumerate}
\item To find out the weekly demand for different sizes of shirt, a shopkeeper kept records of sales of sizes as shown in 
\tabref{tab:shirt}.  This is the record for a week.  Find the mode of the data.
	\begin{table}[H]
  \centering
  \input{tables/shirt.tex}
  \caption{}
  \label{tab:shirt}
\end{table}
\item Find the mode of the given set of numbers
	$$1,1,1,2,2,2,2,3,4,4$$
\end{enumerate}
Find the mode of
\begin{enumerate}[label=\thesubsection.\arabic*, ref=\thesubsection.\theenumi,resume*]
\item 	
		\begin{gather*}
		2,6,5,3,0,4,3,2,4,5,2,4
	\end{gather*}
\item 
		\begin{gather*}
		2,4,16,12,14,14,16,14,10,14,18,14
	\end{gather*}
	\item 
		\begin{gather*}
		2,2,2,3,3,4,5,5,5,6,6,8
	\end{gather*}
	\item 
		\begin{gather*}
		12, 14, 12, 16, 15, 13, 14, 18, 19, 12, 14, 15, 16, 15, 16, 16,
	\\
		15, 17, 13, 16, 16, 15, 15, 13, 15, 17, 15, 14, 15, 13, 15, 14
		\end{gather*}
	\item Heights (in cm) of 25 children given below
		\begin{gather*}
		168, 165, 163, 160, 163, 161, 162, 164, 163, 162,
164, 163,	160, 
		\\
	 163, 160, 165, 163, 162, 163, 164, 163, 160, 165, 163, 162 
	\end{gather*}
		What is the mode of their heights? What do we understand by mode here?
\end{enumerate}
\begin{enumerate}[label=\thesubsection.\arabic*, ref=\thesubsection.\theenumi,resume*]
\item Your friend found the median and the mode of a given data. Describe and correct your friends error if any
\begin{gather*}
	35, 32, 35, 42, 38, 32, 34 
\end{gather*}
Median = 42, Mode = 32.
\item Find the median of the data: 24, 36, 46, 17, 18, 25, 35.
\item The scores in mathematics test (out of 25) of 15 students is as follows
	\begin{gather*}
	19, 25, 23, 20, 9, 20, 15, 10, 5, 16, 25, 20, 24, 12, 20 
\end{gather*}
Find the mode and median of this data. Are they same?
\item The runs scored in a cricket match by 11 players is as follows 
	\begin{gather*}
	6, 15, 120, 50, 100, 80, 10, 15, 8, 10, 15
\end{gather*}
Find the mean, mode and median of this data. Are the three same?
\item The weights (in kg.) of 15 students of a class are
	\begin{gather*}
	38, 42, 35, 37, 45, 50, 32, 43, 43, 40, 36, 38, 43, 38, 47
\end{gather*}
\begin{enumerate}
	\item  Find the mode and median of this data. 
	\item Is there more than one mode?
\end{enumerate}
\item Find the mode and median of the data
	\begin{gather*}
	13, 16, 12, 14, 19, 12, 14, 13, 14
\end{gather*}
\item
	The data 
	\begin{gather*}
	6, 4, 3, 8, 9, 12, 13, 9 
\end{gather*}
has mean 9.  True or False?
\item Two hundred students of 6th and 7th classes were asked to name their favourite colour so as to decide upon what should be the colour of their school building. The results are shown in the following table
  in   \tabref{tab:colour}.
	 Represent the given data on a bar graph
	\begin{table}[H]
  \centering
  \input{tables/colour.tex}
  \caption{}
  \label{tab:colour}
\end{table}
Answer the following questions with the help of the bar graph
	\begin{enumerate}
	\item Which is the most preferred colour and which is the least preferred? 
	\item How many colours are there in all? What are they?
	\end{enumerate}
\item Following data 
  in \tabref{tab:marks}
	gives total marks (out of 600) obtained by six children of a particular class. Represent the data on a bar graph.
	\begin{table}[H]
  \centering
  \input{tables/marks.tex}
  \caption{}
  \label{tab:marks}
\end{table}
\item Consider the following two collections of data 
in
  \tabref{tab:hours}
	giving the average daily hours of sunshine in two cities Aberdeen and Margate for all the twelve months of the year. These cities are near the south pole and hence have only a few hours of sunshine each day.
	In a particular month, which city has more sunshine hours?  Explain through a double bar graph.
	\begin{table}[H]
  \centering
  \input{tables/hours.tex}
  \caption{}
  \label{tab:hours}
\end{table}
\item A mathematics teacher wants to see, whether the new technique of teaching she applied after quarterly test was effective or not. She takes the scores of the 5 weakest children in the quarterly test (out of 25) and in the half yearly test (out of 25) which are listed in 
  \tabref{tab:students}.  Is her technique effective?
	\begin{table}[H]
  \centering
  \input{tables/students.tex}
  \caption{}
  \label{tab:students}
\end{table}
\item 
	Number of children in six different classes are given below
  in \tabref{tab:children}.  
 Represent the data on a bar graph.
	\begin{table}[H]
  \centering
  \input{tables/children.tex}
  \caption{}
  \label{tab:children}
\end{table}
\begin{enumerate}
	\item How would you choose a scale? 
	\item  Which class has the maximum number of children? And the minimum? 
	\item  Find the ratio of students of class sixth to the students of class eight.
\end{enumerate}
\item The performance of a student in 1st Term and 2nd Term is given
in \tabref{tab:term}.
	Draw a double bar graph choosing appropriate scale and answer the following
	\begin{table}[H]
  \centering
  \input{tables/term.tex}
  \caption{}
  \label{tab:term}
\end{table}
\begin{enumerate}
\item In which subject, has the child improved his performance the most? 
\item In which subject is the improvement the least? 
\item Has the performance gone down in any subject?
\end{enumerate}
\item Consider this data 
  in \tabref{tab:sport}
	collected from a survey of a colony.
	\begin{table}[H]
  \centering
  \input{tables/sport.tex}
  \caption{}
  \label{tab:sport}
\end{table}
\begin{enumerate}
\item Draw a double bar graph choosing an appropriate scale. What do you infer from the bar graph?
\item Which sport is most popular? 
\item Which is more preferred, watching or participating in sports?
\end{enumerate}
\end{enumerate}
Write the following rational numbers in ascending order
\begin{enumerate}[label=\thesubsection.\arabic*, ref=\thesubsection.\theenumi,resume*,itemsep=1ex]
	\begin{multicols}{3}
	\item $\frac{-3}{5},\frac{-2}{5},\frac{-1}{5}$
	\item $\frac{-1}{3},\frac{-2}{9},\frac{-4}{3}$
	\item $\frac{-3}{7},\frac{-3}{2},\frac{-3}{4}$
	\end{multicols}
\end{enumerate}
\begin{enumerate}[label=\thesubsection.\arabic*, ref=\thesubsection.\theenumi,resume*]
\item Find
	\begin{enumerate}
	\begin{multicols}{3}
	\item $2.7\times 4$
	\item $1.8\times 1.2$
	\item $2.3\times 4.35$
	\end{multicols}
	\end{enumerate}
and arrange the products in descending order.
\end{enumerate}

\section{Math Library}
\begin{enumerate}[label=\thesection.\arabic*, ref=\thesection.\theenumi]
\item $\triangle ABC$ is right-angled at $C$. If $AC = 5 cm$ and $BC = 12 cm$ find the length of $AB$.
	\\
	\solution
	\lstinputlisting{codes/math/baudh.c}
\item Determine whether the triangle whose lengths of sides are $3 cm, 4 cm, 5 cm$ is a right-angled triangle.
\item $PQR$ is a triangle, right-angled at $P$. If $PQ = 10cm$ and $PR = 24 cm$, find $QR$.
\item $ABC$ is a triangle, right-angled at $C$. If $AB = 25 cm$ and $AC = 7 cm$, find $BC$.
\item A $15 m$ long ladder reached a window $12 m$ high from the ground on placing it against a wall at a distance $a$. Find the distance of the foot of the ladder from the wall.
\item  Which of the following can be the sides of a right triangle? 
\begin{enumerate}
	\item $2.5 cm,6.5 cm, 6 cm.$ 
	\item $ 2 cm, 2 cm, 5 cm.$ 
	\item $ 1.5 cm, 2cm, 2.5 cm.$
\end{enumerate}
\item A tree is broken at a height of $5 m$ from the ground and its top touches the ground at a distance of $12 m$ from the base of the tree. Find the original height of the tree.
\item Find the perimeter of the rectangle whose length is $40 cm$ and a diagonal is $41 cm$. 
\item The diagonals of a rhombus measure $16 cm$ and $30 cm$. Find its perimeter.
\item Find the values of the following expressions for $x = 2$. 
	\begin{enumerate}
\item  $19-5x^2$
\item  $100 – 10x^3$
	\end{enumerate}
\item Find the value of the following expressions when $n = – 2.$ 
	\begin{enumerate}
\item  $5n^2 + 5n – 2 $
\item  $n^3 + 5n^2 + 5n – 2$
	\end{enumerate}
\item Find the value of the following expressions for $a = 3, b = 2$. 
	\begin{enumerate}
\item $ a^2+2ab+b^2$
\item $ a^3 – b^3$
\end{enumerate}
\item If $m = 2$, find the value of
	\begin{enumerate}
\item $ 3m^2 – 2m – 7 $
\item $ \frac{5m^4}{ 2}$
\end{enumerate}
\item  If $p = – 2$, find the value of
	\begin{enumerate}
\item  $– 3p^2 + 4p + 7 $
\item  $– 2p^3 – 3p^2 + 4p + 7$
\end{enumerate}
\item  Find the value of the following expressions, when $x = –1$ 
	\begin{enumerate}
\item $ 2x^2 – x – 2$
\item $ x^2 + 2x +1$
\end{enumerate}
\item  If $a = 2, b = – 2$, find the value of
	\begin{enumerate}
\item  $ a^2 + b^2$
\item  $a^2 + ab + b^2 $
\item  $a^2 – b^2$
\end{enumerate}
\item  When $a = 0, b = – 1$, find the value of the given expressions
	\begin{enumerate}
\item $2a^2 + b^2 + 1 $
\item $2a^2b + 2ab^2 + ab $
\item $a^2 + ab + 2$
\end{enumerate}
\item If $z = 10$, find the value of $z^3 – 3(z – 10)$. 
\item  If $p = – 10$, find the value of $p^2 – 2p – 100$
\item  Simplify the expression and find its value when $a = 5$ and $b = – 3$
	$$ 2a^2 + ab + 3 $$
\item What is the circumference of a circle of diameter 10 cm?
	\\
	\solution
	\lstinputlisting{codes/math/circ.c}
\item What is the circumference of a circular disc of radius 14 cm?
\item The radius of a circular pipe is 10 cm. What length of a tape is required to wrap once around the pipe?
\item Sudhanshu divides a circular disc of radius 7 cm in two equal parts. What is the perimeter of each semicircular shape disc?
\item Find the area of a circle of radius 30 cm?
\item Diameter of a circular garden is 9.8 m. Find its area.
\item Find the circumference of the circles with the following radius  
	\begin{enumerate}
		\item 28 mm
\item  14 cm
\end{enumerate}
\item  Find the area of the following circles, given that the radius is
	\begin{enumerate}
\item 14 mm 
\item 5 cm
\item 21 cm 
\item  diameter = 49 m
\end{enumerate}
\item If the circumference of a circular sheet is 154 m, find its radius. Also find the area of the sheet. 
\item A gardener wants to fence a circular garden of diameter 21m. Find the length of the rope he needs to purchase, if he makes 2 rounds of fence. Also find the cost of the rope, if it costs \rupee 4 per meter. 
\item  From a circular sheet of radius 4 cm, a circle of radius 3 cm is removed. Find the area of the remaining sheet. 
\item Seema wants to put a lace on the edge of a circular table cover of diameter 1.5 m. Find the length of the lace required and also find its cost if one meter of the lace costs
\rupee 15. 
\item Find the cost of polishing a circular table-top of diameter 1.6 m, if the rate of polishing is \rupee $15/m^2$. 
\item Shalya took a wire of length 44 cm and bent it into the shape of a circle. Find the radius of that circle. Also find its area. If the same wire is bent into the shape of a square, what will be the length of each of its sides? Which figure encloses more
area, the circle or the square? 
\item From a circular card sheet of radius 14 cm, two circles of radius 3.5 cm and a rectangle of length 3 cm and breadth 1cm are removed. 
 Find the area of the remaining sheet. 
\item A circle of radius 2 cm is cut out from a square piece of an aluminium sheet of side 6 cm. What is the area of the left over aluminium sheet? 
\item  The circumference of a circle is 31.4 cm. Find the radius and the area of the circle. 
\item A circular flower bed is surrounded by a path 4 m wide. The diameter of the flower bed is 66 m. What is the area of this path? 
\item A circular flower garden has an area of $314 m^2$. A sprinkler at the centre of the garden can cover an area that has a radius of 12 m. Will the sprinkler water the entire garden? 
\item How many times a wheel of radius 28 cm must rotate to go 352 m? 
\item The minute hand of a circular clock is 15 cm.
	How far does the tip of the minute hand move in 1 hour? 
\item The two sides of the parallelogram ABCD are 6 cm and 4 cm. The height corresponding to the base CD is 3 cm. Find the
	\begin{enumerate}
\item 	area of the parallelogram. 
\item the height corresponding to the base AD.
\end{enumerate}
Find the value of
\begin{enumerate}[label=\thesection.\arabic*, ref=\thesection.\theenumi,resume*,itemsep=1ex]
	\begin{multicols}{4}
		\item ${2}\times {10}^{3}$
		\item ${7}^{2}\times {2}^{2}$
		\item ${4}^{3}\times {2}^{3}$
		\item ${5}^{6}\times \brak{-2}^{6}$
		\item $\brak{-2}^{4}\times {-3}^{4}$
		\item ${2}^{3}\times {5}$
		\item ${3}^{}\times {4}^{4}$
		\item ${5}^{2}\times {3}^{3}$
		\item ${2}^{4}\times {3}^{2}$
		\item ${3}^{2}\times {10}^{4}$
		\item $\brak{-3}\times {-2}^{3}$
		\item $\brak{-3}^{2}\times {-5}^{2}$
		\item ${-2}^{3}\times {-10}^{3}$
		\item ${4}^{5}\div{3}^{5}$
		\item ${5}^{6}\div \brak{-2}^{6}$
		\item $\brak{\frac{3^7}{3^2}}\times 3^{5}$
		\item ${2}^{3}\times {2}^{2}\times {5}^{5}$
		\item ${6}^{2}\times {6}^{4}\div {6}^{3}$
		\item ${8}^{2}\div {2}^{3}$
		\item $\brak{2^2}^{3}\times {3}^{6}\times {5}^{6}$
	\item $\frac{\brak{12}^{4}\times {9}^{3}\times {4}}{\brak{6}^{3}\times {8}^{2}\times {27}}$
	\item $\frac{\brak{2}\times {3}^{4}\times {2}^{5}}{{9}\times {4}^{2}}$
		\item ${3}^{2}\times {3}^{4}\times {3}^{8}$
		\item ${6}^{15}\div {6}^{10}$
		\item $\brak{5^2}^{3}\div  {5}^{3}$
		\item ${2}^{5}\times {5}^{5}$
		\item $\brak{3^4}^{3}$
		\item $\brak{{2}^{20}\div{2}^{15}}\times{2}^{3}$
		\item $\frac{\brak{2}^{3}\times {3}^{4}\times {4}^{}}{{3}\times {32}^{}}$
		\item $\brak{{5^2}^{3}\times {5^2}^{}}\div  {5}^{7}$
		\item ${25}^{4}\div  {5}^{3}$
	\item $\frac{\brak{3}\times {7}^{2}\times {11}^{8}}{{21}\times {11}^{3}}$
	\item $\frac{\brak{3}^7}{{3}^{4}\times {3}^{3}}$
		\item $\brak{{2}^{3}\times{2}}^{2}$
		\item $\frac{\brak{2^5}^{2}\times {7}^{3}}{{8}^3\times {7}}$
	\item $\frac{\brak{3}^{5}\times {10}^{5}\times {25}}{\brak{5}^{7}\times {6}^{5}}$
	\end{multicols}
\end{enumerate}
Find the logarithms
\begin{enumerate}[label=\thesection.\arabic*, ref=\thesection.\theenumi,resume*]
	\item 512 base 2
	\\
	\solution
	\lstinputlisting{codes/math/ln.c}
	\begin{multicols}{4}
	\item 256 base 2
	\item  343 base 7
	\item  729 base 3
	\item  3125 base 5
	\end{multicols}
\end{enumerate}
Identify the greater number, wherever possible, in each of the following
\begin{enumerate}[label=\thesection.\arabic*, ref=\thesection.\theenumi,resume*]
	\begin{multicols}{2}
\item $4^3      \text{ or } 3^4$               
\item $ 5^3     \text{ or } 3^5$ 
\item $ 100^2   \text{ or } 2^{100} $ 
\item $2^{10}  \text{ or } 10^2$
\item $2^{3}  \text{ or } 3^{2}$
\item $8^{2}  \text{ or } 2^{8}$
\item $2.7\times 10^{12}  \text{ or } 1.5\times 10^8$
\item $4\times 10^{14}  \text{ or } 3\times 10^{17}$
\end{multicols}
\end{enumerate}
Express each of the following as product of powers of their prime factors
\begin{enumerate}[label=\thesection.\arabic*, ref=\thesection.\theenumi,resume*]
	\begin{multicols}{4}
\item 	648
\item 	405
\item 	540
\item 	3600
\item 	72
\item 	432
\item 	1000
\item 	16000
\item 	270
\item 	768
\item $108 \times 192$
\item $629\times 65$
\end{multicols}
\end{enumerate}
Use ifelse  for the following
\begin{enumerate}[label=\thesection.\arabic*, ref=\thesection.\theenumi,resume*]
\item $10\times 10^{11} = 100^{11}$
\item $2^{3} > 5^{2}$
\item $2^3\times 3^{2} = 6^{5}$
\item $3^0 = 1000^{0}$
\end{enumerate}
\end{enumerate}

\section{Random Numbers}
\begin{enumerate}[label=\thesection.\arabic*, ref=\thesection.\theenumi]
	\item Take a board marked from -104 to 104 as shown in 
	\figref{fig:game1}.
	\item Take a bag containing two blue and two red dice.  Number of dots on the blue dice indicate positive integers and number of dots on the red dice indicate negative integers.
	\item Every player will place his/her counter at zero.
	\item Each player will take out two dice at a time from the bag and throw them.
	\item After every throw, the player has to multiply the numbers marked on the dice.
	\item If the product is a positive integer then the player will move his counter towards 104; if the product is a negative integer then the player will move his counter towards -104.
	\item The player who reaches either -104 or 104 first is the winner.
		\begin{figure}[H]
  \centering
  \includegraphics[width=\columnwidth]{figs/game1.jpg}
  \caption{}
  \label{fig:game1}
\end{figure}
\end{enumerate}
%
\begin{enumerate}[label=\thesection.\arabic*, ref=\thesection.\theenumi,resume*]
	\item Write a program to simulate the game.  Give the inputs manually.
	\\
	\solution 
	\lstinputlisting{codes/rv/game.c}
	\item Revise the program by replacing the second player with the computer.  The computer generates the inputs randomly as follows 
		\begin{enumerate}
			\item Generate the numbers on all the dice using a uniform distribution ranging from 1 to 6.
			\item Simulate the blue and red dice through a Bernoulli distribution having values 1 and -1.
		\end{enumerate}
	\solution 
	\lstinputlisting{codes/rv/rgame.c}
\item Now revise the program so that both players are simulated by the computer.
\end{enumerate}

\newpage
\appendices
\section{Triangle}
%\numberwithin{equation}{section}
Consider a triangle with vertices
		\begin{align}
			\label{eq:app-tri-pts}
			\vec{A} = \myvec{1 \\ -1},\,
			\vec{B} = \myvec{-4 \\ 6},\,
			\vec{C} = \myvec{-3 \\ -5}
		\end{align}
\subsection{Sides}
\input{triangle/vector}
\subsection{Median}
\begin{enumerate}[label=\thesubsection.\arabic*.,ref=\thesubsection.\theenumi]
\numberwithin{equation}{enumi}
\item If $\vec{D}$ divides $BC$ in the ratio $k : 1$,
		\begin{align}
			\vec{D}= \frac{k\vec{C}+\vec{B}}{k+1}
	  \label{eq:app-section_formula}
		\end{align}
		Find the mid points $\vec{D}, \vec{E}, \vec{F}$ of the sides $BC, CA$ and $AB$ respectively.
	\\
		\input{solutions/1/2/1/main.tex}  
	\item Find the centroid  of $\triangle ABC$ which divides $AD$ in the ratio $2:1$.
		\begin{align}
			\vec{G}=\frac{\vec{A}+\vec{B}+\vec{C}}{3}
		\end{align}
   \\
\solution
\begin{equation}
\begin{split}
\label{eq:centroid}
    \vec{G}&= \frac{\myvec{1\\-1}+\myvec{-4\\6}+\myvec{-3\\-5}}{3}\\    
     &= \myvec{-2\\0}
\end{split}
\end{equation}
\begin{lstlisting}
	codes/msoft/medians.c
\end{lstlisting}
\end{enumerate}

\subsection{Altitude}
\input{triangle/altitude}
\subsection{Perpendicular Bisector}
\input{triangle/perp-bisect}
%\vspace{-8mm}
\subsection{Angle Bisector}
\input{triangle/angle-bisect}
\iffalse
\subsection{Eigenvalues and Eigenvectors}
\input{triangle/eigen}
\fi
\end{document}

  
	\item Find the centroid  of $\triangle ABC$ which divides $AD$ in the ratio $2:1$.
		\begin{align}
			\vec{G}=\frac{\vec{A}+\vec{B}+\vec{C}}{3}
		\end{align}
   \\
\solution
\begin{equation}
\begin{split}
\label{eq:centroid}
    \vec{G}&= \frac{\myvec{1\\-1}+\myvec{-4\\6}+\myvec{-3\\-5}}{3}\\    
     &= \myvec{-2\\0}
\end{split}
\end{equation}
\begin{lstlisting}
	codes/msoft/medians.c
\end{lstlisting}
\end{enumerate}

\subsection{Altitude}
\input{triangle/altitude}
\subsection{Perpendicular Bisector}
\input{triangle/perp-bisect}
%\vspace{-8mm}
\subsection{Angle Bisector}
\input{triangle/angle-bisect}
\iffalse
\subsection{Eigenvalues and Eigenvectors}
\input{triangle/eigen}
\fi
\end{document}

