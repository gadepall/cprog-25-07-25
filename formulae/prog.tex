In a quiz, team A scored $a_1 = -40, a_2=10, a_3=0$ and team B scored $b_1=10, b_2=0, b_3=-40$ in three successive rounds.
\begin{enumerate}[label=\thesubsection.\arabic*, ref=\thesubsection.\theenumi]
\item  If the total scores are 
	\begin{align}
		a &= a_1+a_2+a_3
		\\
		b &= b_1+b_2+b_3
	\end{align}
	which team scored more? 
	\\
	\solution 
	\lstinputlisting{codes/prog/ifelse.c}
\item Write a function to compare the final scores.  Check for the cases when $a = -40, b = -40; a = 30, b = 20; a = -20, b = -10$.
	\\
	\solution 
	\lstinputlisting{codes/prog/func.c}
\item Use arrays and a for loop to evaluate 
	\begin{align}
		a &= \sum_{i=0}^{2}a_i
		\\
		b &= \sum_{i=0}^{2}b_i
	\end{align}
	\\
	\solution 
	\lstinputlisting{codes/prog/loop.c}
\item Revise the above code using only functions.
	\\
	\solution 
	\lstinputlisting{codes/prog/loopfunc.c}
\item Use files for the input data.
	\\
	\solution 
	\lstinputlisting{codes/prog/fileprogs/files.c}
\item Revise the files program using pointer arrays
	\\
	\solution 
	\lstinputlisting{codes/prog/fileprogs/pointer.c}
\item Revise the files program using only functions
	\\
	\solution 
	\lstinputlisting{codes/prog/fileprogs/filesfunc.c}
\end{enumerate}
Reduce to standard form
\begin{enumerate}[label=\thesubsection.\arabic*, ref=\thesubsection.\theenumi,resume*,itemsep=1ex]
	\item $\frac{-45}{30}$
		\\
		\solution
	\lstinputlisting{codes/prog/hcf.c}
\end{enumerate}
Use recursion for the following
\begin{enumerate}[label=\thesubsection.\arabic*, ref=\thesubsection.\theenumi,resume*]
		\item $9^3$
			\\
			\solution For integer powers, the exponent can be computed as
			\begin{align}
					x(n) = ax(n-1), x(0) = 1 
			\end{align}
			which results in the following C code.
	\lstinputlisting{codes/prog/recur.c}
\end{enumerate}
Use matrices for the following
\begin{enumerate}[label=\thesubsection.\arabic*, ref=\thesubsection.\theenumi,resume*]
	\item The difference in the measures of two complementary angles is 12\degree. Find the measures of the angles.
		\\
		\solution Let the angles be $x$ and $y$.  Then we have the following equations
		\begin{align}
			x+y = 180
			\\
			x-y = 12
		\end{align}
		which can be expressed as the matrix equation
		\begin{align}
		\myvec{1 & 1 \\ 1 & -1}\myvec{x\\y}= \myvec{180 \\ 12}
		\\
		\text{or, } \vec{A}\vec{x} = \vec{b}
		\end{align}
		The solution can be obtained as
		\begin{align}
			\vec{x} = \frac{\vec{A}^{\top}\vec{b}}{2}
		\end{align}
		using the following code
	\lstinputlisting{codes/prog/mat/matsol.c}
	by keeping the functions in .h files as below.  
	\lstinputlisting{codes/prog/mat/libs/matfun.h}
	\lstinputlisting{codes/prog/mat/libs/geofun.h}
\end{enumerate}
